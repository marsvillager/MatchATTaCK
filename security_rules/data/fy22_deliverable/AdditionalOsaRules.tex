\documentclass[a4paper]{report}

\usepackage{filecontents}
\usepackage[pdftex]{graphicx}
\usepackage{hyperref}

%%% Header and Footer
\usepackage{fancyhdr}
\pagestyle{fancy}
\fancyhf{}
\fancyhead[L]{CONFIDENTIAL}
\fancyhead[R]{\rightmark}
\fancyfoot[C]{\thepage}
\fancyfoot[R]{CONFIDENTIAL}

\usepackage{etoolbox}
\patchcmd{\chapter}{\thispagestyle{plain}}{\thispagestyle{fancy}}{}{}
\patchcmd{\section}{\thispagestyle{plain}}{\thispagestyle{fancy}}{}{}

%%% Chapters & Sections
\usepackage{titlesec}
\titleformat{\chapter}[display]{\normalfont\huge\bfseries}{\chaptertitlename~\thechapter}{20pt}{\Huge}
\titlespacing*{\chapter}{0pt}{70pt}{40pt}

%%% Tables
\usepackage{longtable}
\usepackage{array}
\usepackage{ragged2e}
\newcolumntype{P}[1]{>{\RaggedRight\hspace{0pt}}p{#1}}
\setlength{\extrarowheight}{5pt}

\usepackage{makecell}
\renewcommand{\cellalign}{l}

%%% Images
\usepackage{float}

%%% Captions
\usepackage[font={small}]{caption}

%%% Table of Contents

%%% Appendix
\usepackage[toc,page]{appendix}
\usepackage{pdfpages}

%%% Title
\makeatletter
\renewcommand{\maketitle}{\bgroup\setlength{\parindent}{0pt}

   \begingroup\centering

  {\Huge\textbf{\@title}} \vspace{2ex}

  {\large\@author} \vspace{2ex}

  {\@date} \vspace{2ex}

   \endgroup

}
\makeatother

\title{OSA Rule Creation}
\date{\today}
\author{Siemens AG, T CST}

%%% Extras
\usepackage{url}
\makeatletter
\g@addto@macro{\UrlBreaks}{\UrlOrds}
\makeatother

\usepackage{setspace}

\usepackage{xhfill}
\setlength{\parindent}{0pt}

\usepackage{amssymb}

\usepackage[inline]{enumitem}

\newlist{todolist}{itemize}{2}
\setlist[todolist]{label=$\square$}
\usepackage{pifont}
\newcommand{\cmark}{\ding{51}}%
\newcommand{\xmark}{\ding{55}}%
\newcommand{\done}{\rlap{$\square$}{\raisebox{2pt}{\large\hspace{1pt}\cmark}}%
\hspace{-2.5pt}}
\newcommand{\wontfix}{\rlap{$\square$}{\large\hspace{1pt}\xmark}}

\newcommand{\inlineitem}[1][]{%
\ifnum\enit@type=\tw@
    {\descriptionlabel{#1}}
  \hspace{\labelsep}%
\else
  \ifnum\enit@type=\z@
       \refstepcounter{\@listctr}\fi
    \quad\@itemlabel\hspace{\labelsep}%
\fi}
\makeatother
\parindent=0pt

\showhyphens{createUnspecifiedNodeErrorMarker}

\makeatletter
\def\camelcasenewline{\leavevmode\begingroup
\let\ifcase\iftrue
\def\or##1{%
  \catcode`##1\active\uccode`\~`##1\uppercase{%
    \def~{\egroup\penalty2\hbox\bgroup\string##1}}}%
\@Alph{}%
\@camelcasenewline}

\def\@camelcasenewline#1{\textbf{\hbox\bgroup#1\egroup}\endgroup}
\makeatother

%%% Draft
%\usepackage{showframe}

%\usepackage{draftwatermark}
%\SetWatermarkText{DRAFT}
%\SetWatermarkScale{5}

\usepackage[colorinlistoftodos,textsize=scriptsize]{todonotes}

%%% Json Visualization
\usepackage{listings}
\usepackage{xcolor}

\colorlet{punct}{red!60!black}
\definecolor{background}{HTML}{EEEEEE}
\definecolor{delim}{RGB}{20,105,176}
\colorlet{numb}{magenta!60!black}

\lstdefinelanguage{json}{
    basicstyle=\normalfont\ttfamily,
    numbers=left,
    numberstyle=\scriptsize,
    stepnumber=1,
    numbersep=8pt,
    showstringspaces=false,
    breaklines=true,
    frame=lines,
    backgroundcolor=\color{background},
    literate=
     *{0}{{{\color{numb}0}}}{1}
      {1}{{{\color{numb}1}}}{1}
      {2}{{{\color{numb}2}}}{1}
      {3}{{{\color{numb}3}}}{1}
      {4}{{{\color{numb}4}}}{1}
      {5}{{{\color{numb}5}}}{1}
      {6}{{{\color{numb}6}}}{1}
      {7}{{{\color{numb}7}}}{1}
      {8}{{{\color{numb}8}}}{1}
      {9}{{{\color{numb}9}}}{1}
      {:}{{{\color{punct}{:}}}}{1}
      {,}{{{\color{punct}{,}}}}{1}
      {\{}{{{\color{delim}{\{}}}}{1}
      {\}}{{{\color{delim}{\}}}}}{1}
      {[}{{{\color{delim}{[}}}}{1}
      {]}{{{\color{delim}{]}}}}{1}
}

%%% DOCUMENT %%%
\begin{document}

\begin{figure}
    \begin{flushright}
        \includegraphics[scale=0.2]{images/SiemensLogo.jpg}
        \label{img:siemens_logo}
    \end{flushright}
\end{figure}

\vspace*{5cm}

\maketitle

\newpage

\section*{\center Executive summary}
\markboth{}{}

Initiated in 2016 by T CST SAR-CN, the development process of the ``OT Security Appliance'' (OSA) is currently going through its last productization phase and eventually becoming Siemens's go-to-market solution for industrial monitoring. To support all related activities, T CST has been engaged in a continuous exchange with all OSA's stakeholders (primarily, DI and SMO) to enhance the product (e.g., its features and capabilities) and prepare it for deployment within real infrastructures. This last objective includes the improvement and customization of OSA's ``intelligence'', namely, the rules in charge of detecting cyber attacks against IT and OT systems. OSA correlation rules are, in fact, responsible of targeting on-going anomalous and potentially dangerous behaviors and providing informative alerts to human operators. In this regard, the present document, collects the list of rules specifically crafted in 2022 to monitor SMO products and environments. Such rules are planned to be used in 2023 within two proofs-of-concept projects of SMO: in Vienna (U6 line) and in Frankfurt.

%\clearpage

%\vspace*{8cm}

\tableofcontents

%\clearpage

%%\vspace*{8cm}

%%\chapter{Introduction}
%%\markboth{}{Introduction}
%\addcontentsline{toc}{chapter}{\protect\numberline{}Introduction}%

%\section*{Overview}

%%%\begin{itemize}
%%%  \item Siemens's landscape
%%%  \item Goal of the evaluation
%%%\end{itemize}

%%%\section*{Asset Discovery}

%%%\begin{itemize}
%%%  \item Topic
%%%  \item Current solutions
%%%  \item Limitations
%%%\end{itemize}

%%%\section*{Intrusion Detection}

%%%\begin{itemize}
%%%  \item Topic
%%%  \item Current solutions
%%%  \item Limitations
%%%\end{itemize}

%\clearpage

\vspace*{8cm}

\chapter{New Rule List}
\markboth{}{New Rule List}
%\addcontentsline{toc}{chapter}{\protect\numberline{}Rule List}%

What follows in this chapter is the list of new OSA correlation rules created by T CST in FY22. The new rules are ideally divided in two main groups:
	
\begin{itemize}
	\item {\bf Additional General Rules} (aka ``Windows/Linux Extra Rules'') - integrating the standard set of correlation rules available in OSA 4.5.
	\item {\bf SMO-specific Rules} - targeting specific software used by SMO in customer deployments and thus enhancing OSA capabilities to detect cyber attacks (as well as misbehaviors) in those deployments
\end{itemize}

All rules include the following information:

\begin{itemize}
	\item the ``Categorization'' - indicating whenever the rule can be used for a given operating system
	\item the ``Scope'' - indicating in which SMO environment the rule should be used (information derived from the workshops performed with SMO)
	\item the ``Description'' - providing extra information about the rule as well as possible enhancements for the future
	\item the ``Script'' - representing the actual script to run in the OSA backend
	\item the ``Requirements'' - including all necessary conditions for the rule to run correctly
	\item the ``Parameters'' - including all information to be defined in the script for the rule to run correctly
	\item the ``Additional Information'' - representing all further information and categorization applying to the rule
\end{itemize}

Every rule described in this document links to a corresponding YAML (``.yml'') file including all meta-information needed to a rule to be integrated and loaded in OSA 4.6. The template used to describe each rule is available in the appendix.

\pagebreak

% !TEX root = ../AdditionalOsaRules.tex

%%%%%%%%%
%%% Generic %%%
%%%%%%%%%

%%% Linux
\input{rules/15600_AuditdPasswdCommand}
\input{rules/15601_AuditdUserModification}
\input{rules/15602_AuditdGroupModication}
% !TEX root = ../AdditionalOsaRules.tex

%%%
%
% Version 0.1
%
%%%

\section{Auditd Etc Passwd Modification}
\label{15603}
%\addcontentsline{toc}{section}{\protect\numberline{}(Tool)}%

\openup 1em

\textbf{Full name:} Auditd etc passwd modification\hrulefill \\
%\textbf{Vendor/Provider:} (Vendor name)\hrulefill \\
%\textbf{URL: } \url{http://www.google.de} \hrulefill \\
{\bf Categorization:} \\

\openup -1em
\vspace{-3em}

% Please, write either Multi-OS, Linux, Windows, Mac
\begin{tabular}{p{.5\textwidth}p{.5\textwidth}}

\begin{todolist}
  	\item Multi-OS
	\item[\done] Linux
\end{todolist}
&
\begin{todolist}
	\item Windows
	\item macOS
\end{todolist}

\end{tabular}

\openup 1em

{\bf SMO products in scope:} \\

\openup -1em
\vspace{-3em}

\begin{tabular}{p{.5\textwidth}p{.5\textwidth}}

\begin{todolist}
  \item[\done] Generic
  \item VICOS OC 100
  \item VICOS S\&D
\end{todolist}
&
\begin{todolist}
  \item[\done] Rail9000
  \item[\done] DCS
  \item Others: \hrulefill
\end{todolist}

\end{tabular}

\openup 1em

{\bf Description:} \\

\openup -1em
\vspace{-2em}

%\begin{itemize}
%	\item (What does the rule do?)
%	\item (Why was the situation targeted (e.g., in a workshop)?)
%\end{itemize}

This rule detects the modification of the /etc/passwd file which contains information for all user accounts, i.e. when a user account is modified, added or deleted.

\openup 1em

{\bf Rule query:} \\

\openup -1em
\vspace{-2em}

\begin{lstlisting}[language=json,firstnumber=1]
{
  "query": {
    "bool": {
      "must": [
        {
          "match": {
            "auditdType": "SYSCALL"
          }
        },
        {
          "match": {
            "auditdKey": "etcpasswd"
          }
        },
        {
          "range": {
            "@timestamp": {
              "gte": "<bt>",
              "lt": "<et>"
            }
          }
        }
      ]
    }
  },
  "size": 1000
}
\end{lstlisting}

\openup 1em

{\bf Rule requirements:} \\

\openup -1em
\vspace{-2em}

Auditd should be configured (an example of a suitable configuration has been provided in the appendix). Furthermore, Logstash should be configured to ingest the related data (also in this case, an example of a suitable configuration has been provided in the appendix).

\openup 1em

{\bf Rule parameters:} \\

\openup -1em
\vspace{-2em}

\begin{itemize}
	\item A standard time frame defined by the ``greater than'' (gt) and ``less than'' parameters.
\end{itemize}

\openup 1em

{\bf Further information/categorization:} \\

\openup -1em
\vspace{-2em}

%\begin{itemize}
%	\item <Does the rule have an ATT\&CK label?>
%	\item <Might the rule create many false positives?>
%\end{itemize}

The rule can be labeled with the following MITRE ATT\&CK's tactics and techniques:
\begin{itemize}
	\item T1098 - ``Account Manipulation''
	\item T1136 - ``Create Account''
		\subitem{.001 - `` Local Account''}
	\item T1531 - ``Account Access Removal''
\end{itemize}

\pagebreak


\input{rules/15604_AuditdEtcShadowModification}
\input{rules/15605_AuditdEtcGroupModification}
% !TEX root = ../AdditionalOsaRules.tex

%%%
%
% Version 0.1
%
%%%

\section{Auditd System Time Change}
\label{15606}
%\addcontentsline{toc}{section}{\protect\numberline{}(Tool)}%

\openup 1em

\textbf{Full name:} Auditd system time change\hrulefill \\
%\textbf{Vendor/Provider:} (Vendor name)\hrulefill \\
%\textbf{URL: } \url{http://www.google.de} \hrulefill \\
{\bf Categorization:} \\

\openup -1em
\vspace{-3em}

% Please, write either Multi-OS, Linux, Windows, Mac
\begin{tabular}{p{.5\textwidth}p{.5\textwidth}}

\begin{todolist}
  	\item Multi-OS
	\item[\done] Linux
\end{todolist}
&
\begin{todolist}
	\item Windows
	\item macOS
\end{todolist}

\end{tabular}

\openup 1em

{\bf SMO products in scope:} \\

\openup -1em
\vspace{-3em}

\begin{tabular}{p{.5\textwidth}p{.5\textwidth}}

\begin{todolist}
  \item[\done] Generic
  \item VICOS OC 100
  \item VICOS S\&D
\end{todolist}
&
\begin{todolist}
  \item[\done] Rail9000
  \item[\done] DCS
  \item Others: \hrulefill
\end{todolist}

\end{tabular}

\openup 1em

{\bf Description:} \\

\openup -1em
\vspace{-2em}

%\begin{itemize}
%	\item (What does the rule do?)
%	\item (Why was the situation targeted (e.g., in a workshop)?)
%\end{itemize}

This rule detects system time changes (by the system calls adjtimex and settimeofday)

\openup 1em

{\bf Rule query:} \\

\openup -1em
\vspace{-2em}

\begin{lstlisting}[language=json,firstnumber=1]
{
  "query": {
    "bool": {
      "must": [
        {
          "match": {
            "auditdType": "SYSCALL"
          }
        },
        {
          "match": {
            "auditdKey": "timechange"
          }
        },
        {
          "range": {
            "@timestamp": {
              "gte": "<bt>",
              "lt": "<et>"
            }
          }
        }
      ]
    }
  },
  "size": 1000
}
\end{lstlisting}

\openup 1em

{\bf Rule requirements:} \\

\openup -1em
\vspace{-2em}

Auditd should be configured (an example of a suitable configuration has been provided in the appendix). Furthermore, Logstash should be configured to ingest the related data (also in this case, an example of a suitable configuration has been provided in the appendix).

\openup 1em

{\bf Rule parameters:} \\

\openup -1em
\vspace{-2em}

\begin{itemize}
	\item A standard time frame defined by the ``greater than'' (gt) and ``less than'' parameters.
\end{itemize}

{\bf Further information/categorization:} \\

\openup 1em

{\bf Further information/categorization:} \\

\openup -1em
\vspace{-2em}

%\begin{itemize}
%	\item <Does the rule have an ATT\&CK label?>
%	\item <Might the rule create many false positives?>
%\end{itemize}

None.

\pagebreak


\input{rules/15607_AuditdEffectiveRulesChange}

\input{rules/15699_AccessUnlocked}
% !TEX root = ../AdditionalOsaRules.tex

%%%
%
% Version 0.1
%
%%%

\section{Login Logout At Unusual Time Linux}
\label{15698}
%\addcontentsline{toc}{section}{\protect\numberline{}(Tool)}%

\openup 1em

\textbf{Full name:} Login logout at unusual time linux\hrulefill \\
%\textbf{Vendor/Provider:} (Vendor name)\hrulefill \\
%\textbf{URL: } \url{http://www.google.de} \hrulefill \\
{\bf Categorization:} \\

\openup -1em
\vspace{-3em}

% Please, write either Multi-OS, Linux, Windows, Mac
\begin{tabular}{p{.5\textwidth}p{.5\textwidth}}

\begin{todolist}
  	\item Multi-OS
	\item[\done] Linux
\end{todolist}
&
\begin{todolist}
	\item Windows
	\item macOS
\end{todolist}

\end{tabular}

\openup 1em

{\bf SMO products in scope:} \\

\openup -1em
\vspace{-3em}

\begin{tabular}{p{.5\textwidth}p{.5\textwidth}}

\begin{todolist}
  \item[\done] Generic
  \item VICOS OC 100
  \item VICOS S\&D
\end{todolist}
&
\begin{todolist}
  \item[\done] Rail9000
  \item DCS
  \item Others: \hrulefill
\end{todolist}

\end{tabular}

\openup 1em

{\bf Description:} \\

\openup -1em
\vspace{-2em}

%\begin{itemize}
%	\item (What does the rule do?)
%	\item (Why was the situation targeted (e.g., in a workshop)?)
%\end{itemize}

This rule identifies situations in which Windows login and logout operations are performed or attempted outside the normal working hours (e.g., whenever operators work in shifts, this rule can ensure that such login and logout operations happen in the supposed timeframe). As attackers can try to take advantage of the absence of personnel to exploit the target systems, rules like these one work on restricting the categories of security-related events happening at unexpected time.

\openup 1em

{\bf Rule query:} \\

\openup -1em
\vspace{-2em}

\begin{lstlisting}[language=json,firstnumber=1]
{
  "runtime_mappings": {
    "daytime": {
      "type": "long",
      "script": {
        "source": "emit(doc['@timestamp'].value.toEpochSecond()%86400)"
      }
    }
  },
  "query": {
    "bool": {
      "must": [
        {
          "bool": {
            "should": [
              {
                "match_phrase": {
                  "msg": "session opened for user"
                }
              },
              {
                "match_phrase": {
                  "eventId": "session closed for user"
                }
              }
            ]
          }
        },
       {
          "bool": {
            "must_not": [
              {
                "match_phrase": {
                  "msg": "no new event"
                }
              }
            ]
          }
        },
        {
          "bool": {
            "should": [
              {
                "range": {
                  "daytime": {
                    "lte": "shitfStartingTime"
                  }
                }
              },
              {
                "range": {
                  "daytime": {
                    "gte": "shitfFinishingTime"
                  }
                }
              },
              {
                "range": {
                  "@timestamp": {
                    "gt": "<bt>",
                    "lt": "<et>"
                  }
                }
              }
            ]
          }
        }
      ]
    }
  },
  "fields": [
    "daytime"
  ]
}
\end{lstlisting}

\openup 1em

{\bf Rule requirements:} \\

\openup -1em
\vspace{-2em}

None.

\openup 1em

{\bf Rule parameters:} \\

\openup -1em
\vspace{-2em}

\begin{itemize}
	\item A standard time frame defined by the ``greater than'' (gt) and ``less than'' parameters.
	\item \emph{shitfStartingTime}: The number of seconds from midnights indicating the normal shift starting time (e.g., 28800 would indicate ``8:00'')
	\item \emph{shitfFinishingTime}: The number of seconds from midnights indicating the normal shift ending time (e.g., 64800 would indicate ``18:00'')
\end{itemize}

\openup 1em

{\bf Further information/categorization:} \\

\openup -1em
\vspace{-2em}

%\begin{itemize}
%	\item <Does the rule have an ATT\&CK label?>
%	\item <Might the rule create many false positives?>
%\end{itemize}

The rule can be labeled with the following MITRE ATT\&CK's tactics and techniques:
\begin{itemize}
	\item T1078 - ``Valid Accounts''
		\subitem{.003 - `` Local Accounts''}
\end{itemize}

\pagebreak


\input{rules/15697_ApplicationInstallationLinux}
% !TEX root = ../AdditionalOsaRules.tex

%%%
%
% Version 0.1
%
%%%

\section{Application Removal Linux}
\label{15696}
%\addcontentsline{toc}{section}{\protect\numberline{}(Tool)}%

\openup 1em

\textbf{Full name:} Application removal linux\hrulefill \\
%\textbf{Vendor/Provider:} (Vendor name)\hrulefill \\
%\textbf{URL: } \url{http://www.google.de} \hrulefill \\
{\bf Categorization:} \\

\openup -1em
\vspace{-3em}

% Please, write either Multi-OS, Linux, Windows, Mac
\begin{tabular}{p{.5\textwidth}p{.5\textwidth}}

\begin{todolist}
  	\item Multi-OS
	\item[\done] Linux
\end{todolist}
&
\begin{todolist}
	\item Windows
	\item macOS
\end{todolist}

\end{tabular}

\openup 1em

{\bf SMO products in scope:} \\

\openup -1em
\vspace{-3em}

\begin{tabular}{p{.5\textwidth}p{.5\textwidth}}

\begin{todolist}
  \item[\done] Generic
  \item VICOS OC 100
  \item VICOS S\&D
\end{todolist}
&
\begin{todolist}
  \item Rail9000
  \item DCS
  \item Others: \hrulefill
\end{todolist}

\end{tabular}

\openup 1em

{\bf Description:} \\

\openup -1em
\vspace{-2em}

%\begin{itemize}
%	\item (What does the rule do?)
%	\item (Why was the situation targeted (e.g., in a workshop)?)
%\end{itemize}

This rule identifies situations in which a known application is removed from Linux (via dpkg). Attackers can uninstall applications to compromise the correct functioning of a given system or impair its defense (e.g., uninstalling security tools).

\openup 1em

{\bf Rule query:} \\

\openup -1em
\vspace{-2em}

\begin{lstlisting}[language=json,firstnumber=1]
{
  "query": {
    "bool": {
      "must": [
        {
          "match_phrase": {
            "msg": "remove"
          }
        },
        {
          "match": {
            "facility_label": "user-level"
          }
        },
        {
          "bool": {
            "must_not": [
              {
                "match_phrase": {
                  "msg": "no new event"
                }
              }
            ]
          }
        },
        {
          "range": {
            "@timestamp": {
              "gt": "<bt>",
              "lt": "<et>"
            }
          }
        }
      ]
    }
  },
  "size": 1000
}
\end{lstlisting}

\openup 1em

{\bf Rule requirements:} \\

\openup -1em
\vspace{-2em}

Syslog should be reading and forwarding the ``/var/log/dpkg.log'' file.

\openup 1em

{\bf Rule parameters:} \\

\openup -1em
\vspace{-2em}

\begin{itemize}
	\item A standard time frame defined by the ``greater than'' (gt) and ``less than'' parameters.
\end{itemize}

\openup 1em

{\bf Further information/categorization:} \\

\openup -1em
\vspace{-2em}

None.

\pagebreak


% !TEX root = ../AdditionalOsaRules.tex

%%%
%
% Version 0.1
%
%%%

\section{New Usb Device Plugged In}
\label{15695}
%\addcontentsline{toc}{section}{\protect\numberline{}(Tool)}%

\openup 1em

\textbf{Full name:} New usb device plugged in\hrulefill \\
%\textbf{Vendor/Provider:} (Vendor name)\hrulefill \\
%\textbf{URL: } \url{http://www.google.de} \hrulefill \\
{\bf Categorization:} \\

\openup -1em
\vspace{-3em}

% Please, write either Multi-OS, Linux, Windows, Mac
\begin{tabular}{p{.5\textwidth}p{.5\textwidth}}

\begin{todolist}
  	\item Multi-OS
	\item[\done] Linux
\end{todolist}
&
\begin{todolist}
	\item Windows
	\item macOS
\end{todolist}

\end{tabular}

\openup 1em

{\bf SMO products in scope:} \\

\openup -1em
\vspace{-3em}

\begin{tabular}{p{.5\textwidth}p{.5\textwidth}}

\begin{todolist}
  \item[\done] Generic
  \item VICOS OC 100
  \item VICOS S\&D
\end{todolist}
&
\begin{todolist}
  \item[\done] Rail9000
  \item DCS
  \item Others: \hrulefill
\end{todolist}

\end{tabular}

\openup 1em

{\bf Description:} \\

\openup -1em
\vspace{-2em}

%\begin{itemize}
%	\item (What does the rule do?)
%	\item (Why was the situation targeted (e.g., in a workshop)?)
%\end{itemize}

This rules triggers every time a new external usb device is connected to and recognized by a Linux operating system. Attackers can use external devices to inject malicious code capable of compromising system operation or exfiltrate data. In both cases, any attempt of a system to exchange data whenever unexpected should be alerted right away.

\openup 1em

{\bf Rule query:} \\

\openup -1em
\vspace{-2em}

\begin{lstlisting}[language=json,firstnumber=1]
{
  "query": {
    "bool": {
      "must": [
        {
          "match": {
            "msg": "usb"
          }
        },
        {
          "match": {
            "msg": "new"
          }
        },
        {
          "bool": {
            "must_not": [
              {
                "match_phrase": {
                  "msg": "no new event"
                }
              }
            ]
          }
        },
        {
          "range": {
            "@timestamp": {
              "gt": "<bt>",
              "lt": "<et>"
            }
          }
        }
      ]
    }
  },
  "size": 1000
}
\end{lstlisting}

\openup 1em

{\bf Rule requirements:} \\

\openup -1em
\vspace{-2em}

None.

\openup 1em

{\bf Rule parameters:} \\

\openup -1em
\vspace{-2em}

\begin{itemize}
	\item A standard time frame defined by the ``greater than'' (gt) and ``less than'' parameters.
\end{itemize}

\openup 1em

{\bf Further information/categorization:} \\

\openup -1em
\vspace{-2em}

%\begin{itemize}
%	\item <Does the rule have an ATT\&CK label?>
%	\item <Might the rule create many false positives?>
%\end{itemize}

The rule can be labeled with the following MITRE ATT\&CK's tactics and techniques:
\begin{itemize}
	\item T1200 - ``Hardware Additions''
\end{itemize}

\pagebreak



\input{rules/15650_AuditdGenericFileRead}
\input{rules/15651_AuditdGenericFileWrite}
% !TEX root = ../AdditionalOsaRules.tex

%%%
%
% Version 0.1
%
%%%

\section{Auditd Generic File Execute}
\label{15652}
%\addcontentsline{toc}{section}{\protect\numberline{}(Tool)}%

\openup 1em

\textbf{Full name:} Auditd generic file execute\hrulefill \\
%\textbf{Vendor/Provider:} (Vendor name)\hrulefill \\
%\textbf{URL: } \url{http://www.google.de} \hrulefill \\
{\bf Categorization:} \\

\openup -1em
\vspace{-3em}

% Please, write either Multi-OS, Linux, Windows, Mac
\begin{tabular}{p{.5\textwidth}p{.5\textwidth}}

\begin{todolist}
  	\item Multi-OS
	\item[\done] Linux
\end{todolist}
&
\begin{todolist}
	\item Windows
	\item macOS
\end{todolist}

\end{tabular}

\openup 1em

{\bf SMO products in scope:} \\

\openup -1em
\vspace{-3em}

\begin{tabular}{p{.5\textwidth}p{.5\textwidth}}

\begin{todolist}
  \item[\done] Generic
  \item VICOS OC 100
  \item VICOS S\&D
\end{todolist}
&
\begin{todolist}
  \item Rail9000
  \item DCS
  \item Others: \hrulefill
\end{todolist}

\end{tabular}

\openup 1em

{\bf Description:} \\

\openup -1em
\vspace{-2em}

%\begin{itemize}
%	\item (What does the rule do?)
%	\item (Why was the situation targeted (e.g., in a workshop)?)
%\end{itemize}

This rule detects the execution of a given file.

\openup 1em

{\bf Rule query:} \\

\openup -1em
\vspace{-2em}

\begin{lstlisting}[language=json,firstnumber=1]
{
  "query": {
    "bool": {
      "must": [
        {
          "match": {
            "auditdType": "SYSCALL"
          }
        },
        {
          "match": {
            "auditdKey": "file_x_execute"
          }
        },
        {
          "range": {
            "@timestamp": {
              "gte": "<bt>",
              "lt": "<et>"
            }
          }
        }
      ]
    }
  },
  "size": 1000
}
\end{lstlisting}

\openup 1em

{\bf Rule requirements:} \\

\openup -1em
\vspace{-2em}

Auditd should be configured (an example of a suitable configuration has been provided in the repository). Furthermore, Logstash should be configured to ingest the related data (also in this case, an example of a suitable configuration has been provided in the appendix). A specific path should be included in the auditd rule identified by key ``file\_x\_execute''.

\openup 1em

{\bf Rule parameters:} \\

\openup -1em
\vspace{-2em}

\begin{itemize}
	\item A standard time frame defined by the ``greater than'' (gt) and ``less than'' parameters.
\end{itemize}

\openup 1em

{\bf Further information/categorization:} \\

\openup -1em
\vspace{-2em}

%\begin{itemize}
%	\item <Does the rule have an ATT\&CK label?>
%	\item <Might the rule create many false positives?>
%\end{itemize}

None.

\pagebreak


% !TEX root = ../AdditionalOsaRules.tex

%%%
%
% Version 0.1
%
%%%

\section{Auditd Generic File Remove}
\label{15653}
%\addcontentsline{toc}{section}{\protect\numberline{}(Tool)}%

\openup 1em

\textbf{Full name:} Auditd generic file remove\hrulefill \\
%\textbf{Vendor/Provider:} (Vendor name)\hrulefill \\
%\textbf{URL: } \url{http://www.google.de} \hrulefill \\
{\bf Categorization:} \\

\openup -1em
\vspace{-3em}

% Please, write either Multi-OS, Linux, Windows, Mac
\begin{tabular}{p{.5\textwidth}p{.5\textwidth}}

\begin{todolist}
  	\item Multi-OS
	\item[\done] Linux
\end{todolist}
&
\begin{todolist}
	\item Windows
	\item macOS
\end{todolist}

\end{tabular}

\openup 1em

{\bf SMO products in scope:} \\

\openup -1em
\vspace{-3em}

\begin{tabular}{p{.5\textwidth}p{.5\textwidth}}

\begin{todolist}
  \item[\done] Generic
  \item VICOS OC 100
  \item VICOS S\&D
\end{todolist}
&
\begin{todolist}
  \item[\done] Rail9000
  \item DCS
  \item Others: \hrulefill
\end{todolist}

\end{tabular}

\openup 1em

{\bf Description:} \\

\openup -1em
\vspace{-2em}

%\begin{itemize}
%	\item (What does the rule do?)
%	\item (Why was the situation targeted (e.g., in a workshop)?)
%\end{itemize}

This rule detects the removing action on a given file or directory.

\openup 1em

{\bf Rule query:} \\

\openup -1em
\vspace{-2em}

\begin{lstlisting}[language=json,firstnumber=1]
{
  "query": {
    "bool": {
      "must": [
        {
          "match": {
            "auditdType": "SYSCALL"
          }
        },
        {
          "match": {
            "auditdKey": "file_x_write"
          }
        },
        {
          "match_phrase": {
            "msg": "syscall=263"
          }
        },
        {
          "range": {
            "@timestamp": {
              "gte": "<bt>",
              "lt": "<et>"
            }
          }
        }
      ]
    }
  },
  "size": 1000
}
\end{lstlisting}

\openup 1em

{\bf Rule requirements:} \\

\openup -1em
\vspace{-2em}

Auditd should be configured (an example of a suitable configuration has been provided in the repository). Furthermore, Logstash should be configured to ingest the related data (also in this case, an example of a suitable configuration has been provided in the repository). A specific path should be included in the auditd rule identified by key ``file\_x\_write''.

\openup 1em

{\bf Rule parameters:} \\

\openup -1em
\vspace{-2em}

\begin{itemize}
	\item A standard time frame defined by the ``greater than'' (gt) and ``less than'' parameters.
\end{itemize}

\openup 1em

{\bf Further information/categorization:} \\

\openup -1em
\vspace{-2em}

%\begin{itemize}
%	\item <Does the rule have an ATT\&CK label?>
%	\item <Might the rule create many false positives?>
%\end{itemize}

The rule can be labeled with the following MITRE ATT\&CK's tactics and techniques:
\begin{itemize}
	\item T1485 - ``Data Destruction''
	\item T1565 - ``Data Manipulation''
\end{itemize}

\pagebreak



% !TEX root = ../AdditionalOsaRules.tex

%%%
%
% Version 0.1
%
%%%

\section{Malicious Process Execution Linux}
\label{15660}
%\addcontentsline{toc}{section}{\protect\numberline{}(Tool)}%

\openup 1em

\textbf{Full name:} Malicious process execution linux\hrulefill \\
%\textbf{Vendor/Provider:} (Vendor name)\hrulefill \\
%\textbf{URL: } \url{http://www.google.de} \hrulefill \\
{\bf Categorization:} \\

\openup -1em
\vspace{-3em}

% Please, write either Multi-OS, Linux, Windows, Mac
\begin{tabular}{p{.5\textwidth}p{.5\textwidth}}

\begin{todolist}
  	\item Multi-OS
	\item[\done] Linux
\end{todolist}
&
\begin{todolist}
	\item Windows
	\item macOS
\end{todolist}

\end{tabular}

\openup 1em

{\bf SMO products in scope:} \\

\openup -1em
\vspace{-3em}

\begin{tabular}{p{.5\textwidth}p{.5\textwidth}}

\begin{todolist}
  \item[\done] Generic
  \item VICOS OC 100
  \item VICOS S\&D
\end{todolist}
&
\begin{todolist}
  \item[\done] Rail9000
  \item DCS
  \item Others: \hrulefill
\end{todolist}

\end{tabular}

\openup 1em

{\bf Description:} \\

\openup -1em
\vspace{-2em}

%\begin{itemize}
%	\item (What does the rule do?)
%	\item (Why was the situation targeted (e.g., in a workshop)?)
%\end{itemize}

This rule identifies situations in which a Linux system executes a known malicious process. This event might occur especially in the initial phases of an attack (e.g., attacker running a given malware toolkits to acquire privileges) but it can generally happen in any phase of a cyber intrusion. In most cases, denylists can include folder names from where no files are supposed to be executed.

\openup 1em

{\bf Rule query:} \\

\openup -1em
\vspace{-2em}

\begin{lstlisting}[language=json,firstnumber=1]
{
  "query": {
    "bool": {
      "must": [
        {
          "match": {
            "auditdType": "EXECVE"
          }
        },
        {
          "terms": {
            "msg": [
              "executable1",
              "executable2",
              "..."
            ]
          }
        },
        {
          "range": {
            "@timestamp": {
              "gt": "<bt>",
              "lt": "<et>"
            }
          }
        }
      ]
    }
  },
  "size": 1000
}
\end{lstlisting}

\openup 1em

{\bf Rule requirements:} \\

\openup -1em
\vspace{-2em}

Auditd should be configured (an example of a suitable configuration has been provided in the repository). Furthermore, Logstash should be configured to ingest the related data (also in this case, an example of a suitable configuration has been provided in the repository).

\openup 1em

{\bf Rule parameters:} \\

\openup -1em
\vspace{-2em}

\begin{itemize}
	\item A standard time frame defined by the ``greater than'' (gt) and ``less than'' parameters.
	\item \emph{executableX}: A denylisted process name
\end{itemize}

\openup 1em

{\bf Further information/categorization:} \\

\openup -1em
\vspace{-2em}

%\begin{itemize}
%	\item <Does the rule have an ATT\&CK label?>
%	\item <Might the rule create many false positives?>
%\end{itemize}

The rule can be labeled with the following MITRE ATT\&CK's tactics and techniques:
\begin{itemize}
	\item T1204 - ``User Execution''
		\subitem{.002 - `` Malicious File''}
\end{itemize}

\pagebreak


\input{rules/15662_AnomalousProcessExecutionWithHighPrivilegesLinux}

\input{rules/15670_OsUserAccountCreationLinux}
\input{rules/15671_OsHighPrivilegeUserAccountCreationLinux}
% !TEX root = ../AdditionalOsaRules.tex

%%%
%
% Version 0.1
%
%%%

\section{Os Temporary User Account Linux}
\label{15672}
%\addcontentsline{toc}{section}{\protect\numberline{}(Tool)}%

\openup 1em

\textbf{Full name:} Os temporary user account linux\hrulefill \\
%\textbf{Vendor/Provider:} (Vendor name)\hrulefill \\
%\textbf{URL: } \url{http://www.google.de} \hrulefill \\
{\bf Categorization:} \\

\openup -1em
\vspace{-3em}

% Please, write either Multi-OS, Linux, Windows, Mac
\begin{tabular}{p{.5\textwidth}p{.5\textwidth}}

\begin{todolist}
  	\item Multi-OS
	\item[\done] Linux
\end{todolist}
&
\begin{todolist}
	\item Windows
	\item macOS
\end{todolist}

\end{tabular}

\openup 1em

{\bf SMO products in scope:} \\

\openup -1em
\vspace{-3em}

\begin{tabular}{p{.5\textwidth}p{.5\textwidth}}

\begin{todolist}
  \item[\done] Generic
  \item VICOS OC 100
  \item VICOS S\&D
\end{todolist}
&
\begin{todolist}
  \item[\done] Rail9000
  \item DCS
  \item Others: \hrulefill
\end{todolist}

\end{tabular}

\openup 1em

{\bf Description:} \\

\openup -1em
\vspace{-2em}

%\begin{itemize}
%	\item (What does the rule do?)
%	\item (Why was the situation targeted (e.g., in a workshop)?)
%\end{itemize}

This rules triggers every time an account is removed from within a Linux operating system. Attackers can remove accounts to delete traces of their actions (if the account was owned) or simply make it unavailable for regular users. In stable systems (e.g., servers) managed by a few users any removed account should be alerted. Depending on the criticality of the monitored systems, normal systems might treat the same situation as warnings or events.

\openup 1em

{\bf Rule query:} \\

\openup -1em
\vspace{-2em}

\begin{lstlisting}[language=json,firstnumber=1]
{
  "query": {
    "bool": {
      "must": [
        {
          "match": {
            "program": "userdel"
          }
        },
        {
          "match": {
            "msg": "delete"
          }
        },
        {
          "bool": {
            "must_not": [
              {
                "match_phrase": {
                  "msg": "no new event"
                }
              }
            ]
          }
        },
        {
          "range": {
            "@timestamp": {
              "gt": "<bt>",
              "lt": "<et>"
            }
          }
        }
      ]
    }
  },
  "size": 1000
}
\end{lstlisting}

\openup 1em

{\bf Rule requirements:} \\

\openup -1em
\vspace{-2em}

None.

\openup 1em

{\bf Rule parameters:} \\

\openup -1em
\vspace{-2em}

\begin{itemize}
	\item A standard time frame defined by the ``greater than'' (gt) and ``less than'' parameters.
\end{itemize}

\openup 1em

{\bf Further information/categorization:} \\

\openup -1em
\vspace{-2em}

%\begin{itemize}
%	\item <Does the rule have an ATT\&CK label?>
%	\item <Might the rule create many false positives?>
%\end{itemize}

The rule can be labeled with the following MITRE ATT\&CK's tactics and techniques:
\begin{itemize}
	\item T1531 - ``Account Access Removal''
\end{itemize}

\pagebreak



%%% Windows
% !TEX root = ../AdditionalOsaRules.tex

%%%
%
% Version 0.1
%
%%%

\section{Event Logging Service Shut Down}
\label{15001}
%\addcontentsline{toc}{section}{\protect\numberline{}(Tool)}%

\openup 1em

\textbf{Full name:} Event logging service shut down\hrulefill \\
%\textbf{Vendor/Provider:} (Vendor name)\hrulefill \\
%\textbf{URL: } \url{http://www.google.de} \hrulefill \\
{\bf Categorization:} \\

\openup -1em
\vspace{-3em}

% Please, write either Multi-OS, Linux, Windows, Mac
\begin{tabular}{p{.5\textwidth}p{.5\textwidth}}

\begin{todolist}
  	\item Multi-OS
	\item Linux
\end{todolist}
&
\begin{todolist}
	\item[\done] Windows
	\item macOS
\end{todolist}

\end{tabular}

\openup 1em

{\bf SMO products in scope:} \\

\openup -1em
\vspace{-3em}

\begin{tabular}{p{.5\textwidth}p{.5\textwidth}}

\begin{todolist}
  \item[\done] Generic
  \item VICOS OC 100
  \item VICOS S\&D
\end{todolist}
&
\begin{todolist}
  \item Rail9000
  \item DCS
  \item Others: \hrulefill
\end{todolist}

\end{tabular}

\openup 1em

{\bf Description:} \\

\openup -1em
\vspace{-2em}

%\begin{itemize}
%	\item (What does the rule do?)
%	\item (Why was the situation targeted (e.g., in a workshop)?)
%\end{itemize}

This rule identifies situations in which Windows logging service has shut down. The related issue might be caused voluntarily (e.g., by an attacker trying to hide its actions) or by mistake. In both cases, it is of paramount importance for an operator to intervene and recover the service to avoid any loss of visibility in the machine under analysis.

\openup 1em

{\bf Rule query:} \\

\openup -1em
\vspace{-2em}

\begin{lstlisting}[language=json,firstnumber=1]
{
  "query": {
    "bool": {
      "must": [
        {
          "match": {
            "eventId": "1100"
          }
        },
        {
          "match": {
            "cat": "winEvent"
          }
        },
        {
          "range": {
            "@timestamp": {
              "gt": "<bt>",
              "lt": "<et>"
            }
          }
        }
      ]
    }
  },
  "size": 1000
}
\end{lstlisting}

\openup 1em

{\bf Rule requirements:} \\

\openup -1em
\vspace{-2em}

None.

\openup 1em

{\bf Rule parameters:} \\

\openup -1em
\vspace{-2em}

\begin{itemize}
	\item A standard time frame defined by the ``greater than'' (gt) and ``less than'' parameters.
\end{itemize}

\openup 1em

{\bf Further information/categorization:} \\

\openup -1em
\vspace{-2em}

%\begin{itemize}
%	\item <Does the rule have an ATT\&CK label?>
%	\item <Might the rule create many false positives?>
%\end{itemize}

The rule can be labeled with the following MITRE ATT\&CK's tactics and techniques:
\begin{itemize}
	\item T1562 - ``Impair Defenses''
		\subitem{.002 - `` Disable Windows Event Logging''}
\end{itemize}

\pagebreak


\input{rules/15002_EventLoggingServiceError}

% !TEX root = ../AdditionalOsaRules.tex

%%%
%
% Version 0.1
%
%%%

\section{Authentication Package Loaded}
\label{15010}
%\addcontentsline{toc}{section}{\protect\numberline{}(Tool)}%

\openup 1em

\textbf{Full name:} Authentication package loaded\hrulefill \\
%\textbf{Vendor/Provider:} (Vendor name)\hrulefill \\
%\textbf{URL: } \url{http://www.google.de} \hrulefill \\
{\bf Categorization:} \\

\openup -1em
\vspace{-3em}

% Please, write either Multi-OS, Linux, Windows, Mac
\begin{tabular}{p{.5\textwidth}p{.5\textwidth}}

\begin{todolist}
  	\item Multi-OS
	\item Linux
\end{todolist}
&
\begin{todolist}
	\item[\done] Windows
	\item macOS
\end{todolist}

\end{tabular}

\openup 1em

{\bf SMO products in scope:} \\

\openup -1em
\vspace{-3em}

\begin{tabular}{p{.5\textwidth}p{.5\textwidth}}

\begin{todolist}
  \item[\done] Generic
  \item VICOS OC 100
  \item VICOS S\&D
\end{todolist}
&
\begin{todolist}
  \item Rail9000
  \item DCS
  \item Others: \hrulefill
\end{todolist}

\end{tabular}

\openup 1em

{\bf Description:} \\

\openup -1em
\vspace{-2em}

%\begin{itemize}
%	\item (What does the rule do?)
%	\item (Why was the situation targeted (e.g., in a workshop)?)
%\end{itemize}

This rule identifies all situations where Windows ``Local Security Authority'' (LSA) loads authentication packages from dynamic-link libraries. An attacker might exploit this mechanism to inject malicious code into the given machine forcing privilege escalation or ensuring persistence. .

\openup 1em

{\bf Rule query:} \\

\openup -1em
\vspace{-2em}

\begin{lstlisting}[language=json,firstnumber=1]
{
  "query": {
    "bool": {
      "must": [
        {
          "match": {
            "eventId": "1108"
          }
        },
        {
          "bool": {
            "must_not": [
              {
                "match_phrase": {
                  "msg": "<allowlisted authentication package>"
                }
              }
            ]
          }
        },
        {
          "match": {
            "cat": "winEvent"
          }
        },
        {
          "range": {
            "@timestamp": {
              "gt": "<bt>",
              "lt": "<et>"
            }
          }
        }
      ]
    }
  },
  "size": 1000
}
\end{lstlisting}

\openup 1em

{\bf Rule requirements:} \\

\openup -1em
\vspace{-2em}

None.

\openup 1em

{\bf Rule parameters:} \\

\openup -1em
\vspace{-2em}

\begin{itemize}
	\item A standard time frame defined by the ``greater than'' (gt) and ``less than'' parameters.
	\item \emph{allowlisted authentication package}: the name of the authentication package used by the given Windows version (e.g., ``C:\textbackslash Windows\textbackslash system32\textbackslash msv1\_0.DLL : MICROSOFT\_AUTHENTICATION\_PACKAGE\_V1\_0'' for Windows 10)
\end{itemize}

\openup 1em

{\bf Further information/categorization:} \\

\openup -1em
\vspace{-2em}

%\begin{itemize}
%	\item <Does the rule have an ATT\&CK label?>
%	\item <Might the rule create many false positives?>
%\end{itemize}

The rule can be labeled with the following MITRE ATT\&CK's tactics and techniques:
\begin{itemize}
	\item T1547 - ``Boot or Logon Autostart Execution''
\end{itemize}

\pagebreak


\input{rules/15011_AttemptedLogin}
\input{rules/15012_LoginWithExplicitCredentials}
% !TEX root = ../AdditionalOsaRules.tex

%%%
%
% Version 0.1
%
%%%

\section{Scheduled Task Created}
\label{15013}
%\addcontentsline{toc}{section}{\protect\numberline{}(Tool)}%

\openup 1em

\textbf{Full name:} Scheduled task created\hrulefill \\
%\textbf{Vendor/Provider:} (Vendor name)\hrulefill \\
%\textbf{URL: } \url{http://www.google.de} \hrulefill \\
{\bf Categorization:} \\

\openup -1em
\vspace{-3em}

% Please, write either Multi-OS, Linux, Windows, Mac
\begin{tabular}{p{.5\textwidth}p{.5\textwidth}}

\begin{todolist}
  	\item Multi-OS
	\item Linux
\end{todolist}
&
\begin{todolist}
	\item[\done] Windows
	\item macOS
\end{todolist}

\end{tabular}

\openup 1em

{\bf SMO products in scope:} \\

\openup -1em
\vspace{-3em}

\begin{tabular}{p{.5\textwidth}p{.5\textwidth}}

\begin{todolist}
  \item[\done] Generic
  \item VICOS OC 100
  \item VICOS S\&D
\end{todolist}
&
\begin{todolist}
  \item Rail9000
  \item DCS
  \item Others: \hrulefill
\end{todolist}

\end{tabular}

\openup 1em

{\bf Description:} \\

\openup -1em
\vspace{-2em}

%\begin{itemize}
%	\item (What does the rule do?)
%	\item (Why was the situation targeted (e.g., in a workshop)?)
%\end{itemize}

This rule is used to identify situations in which an attacker (or a malicious program) creates a new task (e.g., to ensure persistence within a given system). Particular cases in which this event occurs might be further specified to avoid the presence of false positives. For example, it is worth noting the opportunity to alert situations in which the event comes with the ``logon type'' set to ``Password''. In this case, the password for the account running the scheduled task is also stored in Windows ``Credential Manager'' in cleartext and can be therefore easiliy retrived.

\openup 1em

{\bf Rule query:} \\

\openup -1em
\vspace{-2em}

\begin{lstlisting}[language=json,firstnumber=1]
{
  "query": {
    "bool": {
      "must": [
        {
          "match": {
            "eventId": "4698"
          }
        },
        {
          "match": {
            "cat": "winEvent"
          }
        },
        {
          "range": {
            "@timestamp": {
              "gt": "<bt>",
              "lt": "<et>"
            }
          }
        }
      ]
    }
  },
  "size": 1000
}
\end{lstlisting}

\openup 1em

{\bf Rule requirements:} \\

\openup -1em
\vspace{-2em}

None.

\openup 1em

{\bf Rule parameters:} \\

\openup -1em
\vspace{-2em}

\begin{itemize}
	\item A standard time frame defined by the ``greater than'' (gt) and ``less than'' parameters.
\end{itemize}

\openup 1em

{\bf Further information/categorization:} \\

\openup -1em
\vspace{-2em}

%\begin{itemize}
%	\item <Does the rule have an ATT\&CK label?>
%	\item <Might the rule create many false positives?>
%\end{itemize}

The rule can be labeled with the following MITRE ATT\&CK's tactics and techniques:
\begin{itemize}
	\item T1053 - ``Scheduled Task/Job''
		\subitem{.005 - `` Scheduled Task''}
\end{itemize}

\pagebreak


% !TEX root = ../AdditionalOsaRules.tex

%%%
%
% Version 0.1
%
%%%

\section{Scheduled Task Enabled}
\label{15014}
%\addcontentsline{toc}{section}{\protect\numberline{}(Tool)}%

\openup 1em

\textbf{Full name:} Scheduled task enabled\hrulefill \\
%\textbf{Vendor/Provider:} (Vendor name)\hrulefill \\
%\textbf{URL: } \url{http://www.google.de} \hrulefill \\
{\bf Categorization:} \\

\openup -1em
\vspace{-3em}

% Please, write either Multi-OS, Linux, Windows, Mac
\begin{tabular}{p{.5\textwidth}p{.5\textwidth}}

\begin{todolist}
  	\item Multi-OS
	\item Linux
\end{todolist}
&
\begin{todolist}
	\item[\done] Windows
	\item macOS
\end{todolist}

\end{tabular}

\openup 1em

{\bf SMO products in scope:} \\

\openup -1em
\vspace{-3em}

\begin{tabular}{p{.5\textwidth}p{.5\textwidth}}

\begin{todolist}
  \item[\done] Generic
  \item VICOS OC 100
  \item VICOS S\&D
\end{todolist}
&
\begin{todolist}
  \item Rail9000
  \item DCS
  \item Others: \hrulefill
\end{todolist}

\end{tabular}

\openup 1em

{\bf Description:} \\

\openup -1em
\vspace{-2em}

%\begin{itemize}
%	\item (What does the rule do?)
%	\item (Why was the situation targeted (e.g., in a workshop)?)
%\end{itemize}

This rule is used to identify situations in which an attacker (or a malicious program) enable a task that would otherwise be disabled in the system. The criticality of this event might be further assessed by looking at the task name or in correspondence with the ``Scheduled Task Created'' rule.

\openup 1em

{\bf Rule query:} \\

\openup -1em
\vspace{-2em}

\begin{lstlisting}[language=json,firstnumber=1]
{
  "query": {
    "bool": {
      "must": [
        {
          "match": {
            "eventId": "4700"
          }
        },
        {
          "match": {
            "cat": "winEvent"
          }
        },
        {
          "range": {
            "@timestamp": {
              "gt": "<bt>",
              "lt": "<et>"
            }
          }
        }
      ]
    }
  },
  "size": 1000
}
\end{lstlisting}

\openup 1em

{\bf Rule requirements:} \\

\openup -1em
\vspace{-2em}

None.

\openup 1em

{\bf Rule parameters:} \\

\openup -1em
\vspace{-2em}

\begin{itemize}
	\item A standard time frame defined by the ``greater than'' (gt) and ``less than'' parameters.
\end{itemize}

\openup 1em

{\bf Further information/categorization:} \\

\openup -1em
\vspace{-2em}

%\begin{itemize}
%	\item <Does the rule have an ATT\&CK label?>
%	\item <Might the rule create many false positives?>
%\end{itemize}

The rule can be labeled with the following MITRE ATT\&CK's tactics and techniques:
\begin{itemize}
	\item T1053 - ``Scheduled Task/Job''
		\subitem{.005 - `` Scheduled Task''}
\end{itemize}

\pagebreak


\input{rules/15015_GroupCreated}
% !TEX root = ../AdditionalOsaRules.tex

%%%
%
% Version 0.1
%
%%%

\section{User Added To Group}
\label{15016}
%\addcontentsline{toc}{section}{\protect\numberline{}(Tool)}%

\openup 1em

\textbf{Full name:} User added to group\hrulefill \\
%\textbf{Vendor/Provider:} (Vendor name)\hrulefill \\
%\textbf{URL: } \url{http://www.google.de} \hrulefill \\
{\bf Categorization:} \\

\openup -1em
\vspace{-3em}

% Please, write either Multi-OS, Linux, Windows, Mac
\begin{tabular}{p{.5\textwidth}p{.5\textwidth}}

\begin{todolist}
  	\item Multi-OS
	\item Linux
\end{todolist}
&
\begin{todolist}
	\item[\done] Windows
	\item macOS
\end{todolist}

\end{tabular}

\openup 1em

{\bf SMO products in scope:} \\

\openup -1em
\vspace{-3em}

\begin{tabular}{p{.5\textwidth}p{.5\textwidth}}

\begin{todolist}
  \item[\done] Generic
  \item VICOS OC 100
  \item VICOS S\&D
\end{todolist}
&
\begin{todolist}
  \item Rail9000
  \item DCS
  \item Others: \hrulefill
\end{todolist}

\end{tabular}

\openup 1em

{\bf Description:} \\

\openup -1em
\vspace{-2em}

%\begin{itemize}
%	\item (What does the rule do?)
%	\item (Why was the situation targeted (e.g., in a workshop)?)
%\end{itemize}

This rule identifies a possible follow-up of the ``User creation'' event. In this scenario, an attacker might try to add a newly created group to a local, a global or a universal group.

\openup 1em

{\bf Rule query:} \\

\openup -1em
\vspace{-2em}

\begin{lstlisting}[language=json,firstnumber=1]
{
  "query": {
    "bool": {
      "must": [
        {
          "bool": {
            "should": [
              {
                "match": {
                  "eventId": "4756"
                }
              },
              {
                "match": {
                  "eventId": "4728"
                }
              },
              {
                "match": {
                  "eventId": "4732"
                }
              }
            ]
          }
        },
        {
          "match": {
            "cat": "winEvent"
          }
        },
        {
          "range": {
            "@timestamp": {
              "gt": "<bt>",
              "lt": "<et>"
            }
          }
        }
      ]
    }
  },
  "size": 1000
}
\end{lstlisting}

\openup 1em

{\bf Rule requirements:} \\

\openup -1em
\vspace{-2em}

None.

\openup 1em

{\bf Rule parameters:} \\

\openup -1em
\vspace{-2em}

\begin{itemize}
	\item A standard time frame defined by the ``greater than'' (gt) and ``less than'' parameters.
\end{itemize}

\openup 1em

{\bf Further information/categorization:} \\

\openup -1em
\vspace{-2em}

%\begin{itemize}
%	\item <Does the rule have an ATT\&CK label?>
%	\item <Might the rule create many false positives?>
%\end{itemize}

The rule can be labeled with the following MITRE ATT\&CK's tactics and techniques:
\begin{itemize}
	\item None assigned
\end{itemize}

\openup 1em

{\bf Further information/categorization:} \\

\openup -1em
\vspace{-2em}

%\begin{itemize}
%	\item <Does the rule have an ATT\&CK label?>
%	\item <Might the rule create many false positives?>
%\end{itemize}

None.

\pagebreak


\input{rules/15017_AccountLockedOut}
\input{rules/15018_UserAccountUnlocked}
% !TEX root = ../AdditionalOsaRules.tex

%%%
%
% Version 0.1
%
%%%

\section{Membership Enumerated}
%\addcontentsline{toc}{section}{\protect\numberline{}(Tool)}%

\openup 1em

\textbf{Full name:} Membership enumerated\hrulefill \\
%\textbf{Vendor/Provider:} (Vendor name)\hrulefill \\
%\textbf{URL: } \url{http://www.google.de} \hrulefill \\
{\bf Categorization:} \\

\openup -1em
\vspace{-3em}

% Please, write either Multi-OS, Linux, Windows, Mac
\begin{tabular}{p{.5\textwidth}p{.5\textwidth}}

\begin{todolist}
  	\item Multi-OS
	\item Linux
\end{todolist}
&
\begin{todolist}
	\item[\done] Windows
	\item macOS
\end{todolist}

\end{tabular}

\openup 1em

{\bf SMO products in scope:} \\

\openup -1em
\vspace{-3em}

\begin{tabular}{p{.5\textwidth}p{.5\textwidth}}

\begin{todolist}
  \item[\done] Generic
  \item VICOS OC 100
  \item VICOS S\&D
\end{todolist}
&
\begin{todolist}
  \item Rail9000
  \item DCS
  \item Others: \hrulefill
\end{todolist}

\end{tabular}

\openup 1em

{\bf Description:} \\

\openup -1em
\vspace{-2em}

%\begin{itemize}
%	\item (What does the rule do?)
%	\item (Why was the situation targeted (e.g., in a workshop)?)
%\end{itemize}

 This rule focuses on situations in which an attacker, oncer gained access to an account, tries to check how valuable such account is. The monitoring of the ``membership enumeration''-like events can be refined by further checking the ``Process Name'' field as, for example, this not being included in the standard folder (e.g., ``system32'') or specifically matching a denylist of widely-used malicious program (e.g., ``mimikatz'', ``cain.exe'', etc.). In this implemention, folders ``System32'' and ``ImmersiveControlPanel'' have been whitelisted together with process ``explorer.exe''

\openup 1em

{\bf Rule query:} \\

\openup -1em
\vspace{-2em}

\begin{lstlisting}[language=json,firstnumber=1]
{
  "query": {
    "bool": {
      "must": [
        {
          "bool": {
            "should": [
              {
                "match": {
                  "eventId": "4798"
                }
              },
              {
                "match": {
                  "eventId": "4799"
                }
              }
            ]
          }
        },
        {
          "bool": {
            "must_not": [
              {
                "match_phrase": {
                  "event_data.CallerProcessName": "C:\\Windows\\System32\\"
                }
              },
              {
                "match_phrase": {
                  "event_data.CallerProcessName": "C:\\Windows\\ImmersiveControlPanel\\"
                }
              },
              {
                "match_phrase": {
                  "event_data.CallerProcessName": "C:\\Windows\\explorer.exe"
                }
              }
            ]
          }
        },
        {
          "match": {
            "cat": "winEvent"
          }
        }
      ]
    }
  },
  "size": 1000
}
\end{lstlisting}

\openup 1em

{\bf Rule requirements:} \\

\openup -1em
\vspace{-2em}

None.

\openup 1em

{\bf Rule parameters:} \\

\openup -1em
\vspace{-2em}

\begin{itemize}
	\item A standard time frame defined by the ``greater than'' (gt) and ``less than'' parameters.
\end{itemize}

\openup 1em

{\bf Further information/categorization:} \\

\openup -1em
\vspace{-2em}

%\begin{itemize}
%	\item <Does the rule have an ATT\&CK label?>
%	\item <Might the rule create many false positives?>
%\end{itemize}

The rule can be labeled with the following MITRE ATT\&CK's tactics and techniques:
\begin{itemize}
	\item T1087 - ``Account Discovery''
		\subitem{.001 - `` Local Account''}
	\item T1087 - ``Account Discovery''
		\subitem{.002 - `` Domain Account''}
\end{itemize}

\pagebreak


% !TEX root = ../AdditionalOsaRules.tex

%%%
%
% Version 0.1
%
%%%

\section{Password Hash Accessed}
\label{15020}
%\addcontentsline{toc}{section}{\protect\numberline{}(Tool)}%

\openup 1em

\textbf{Full name:} Password hash accessed\hrulefill \\
%\textbf{Vendor/Provider:} (Vendor name)\hrulefill \\
%\textbf{URL: } \url{http://www.google.de} \hrulefill \\
{\bf Categorization:} \\

\openup -1em
\vspace{-3em}

% Please, write either Multi-OS, Linux, Windows, Mac
\begin{tabular}{p{.5\textwidth}p{.5\textwidth}}

\begin{todolist}
  	\item Multi-OS
	\item Linux
\end{todolist}
&
\begin{todolist}
	\item[\done] Windows
	\item macOS
\end{todolist}

\end{tabular}

\openup 1em

{\bf SMO products in scope:} \\

\openup -1em
\vspace{-3em}

\begin{tabular}{p{.5\textwidth}p{.5\textwidth}}

\begin{todolist}
  \item[\done] Generic
  \item VICOS OC 100
  \item VICOS S\&D
\end{todolist}
&
\begin{todolist}
  \item Rail9000
  \item DCS
  \item Others: \hrulefill
\end{todolist}

\end{tabular}

\openup 1em

{\bf Description:} \\

\openup -1em
\vspace{-2em}

%\begin{itemize}
%	\item (What does the rule do?)
%	\item (Why was the situation targeted (e.g., in a workshop)?)
%\end{itemize}

This rule fires an alert whenever a password hash of a Windows account is accessed. Unless planned, any access of critical files (such as the one containing account information) should be monitored at all time.

\openup 1em

{\bf Rule query:} \\

\openup -1em
\vspace{-2em}

\begin{lstlisting}[language=json,firstnumber=1]
{
  "query": {
    "bool": {
      "must": [
        {
          "match": {
            "eventId": "4782"
          }
        },
        {
          "match": {
            "cat": "winEvent"
          }
        },
        {
          "range": {
            "@timestamp": {
              "gt": "<bt>",
              "lt": "<et>"
            }
          }
        }
      ]
    }
  },
  "size": 1000
}
\end{lstlisting}

\openup 1em

{\bf Rule requirements:} \\

\openup -1em
\vspace{-2em}

None.

\openup 1em

{\bf Rule parameters:} \\

\openup -1em
\vspace{-2em}

\begin{itemize}
	\item A standard time frame defined by the ``greater than'' (gt) and ``less than'' parameters.
\end{itemize}

\openup 1em

{\bf Further information/categorization:} \\

\openup -1em
\vspace{-2em}

%\begin{itemize}
%	\item <Does the rule have an ATT\&CK label?>
%	\item <Might the rule create many false positives?>
%\end{itemize}

The rule can be labeled with the following MITRE ATT\&CK's tactics and techniques:
\begin{itemize}
	\item T1555 - ``Credentials from Password Stores''
\end{itemize}

\pagebreak


% !TEX root = ../AdditionalOsaRules.tex

%%%
%
% Version 0.1
%
%%%

\section{Login Of Non Registered Account}
\label{15021}
%\addcontentsline{toc}{section}{\protect\numberline{}(Tool)}%

\openup 1em

\textbf{Full name:} Login of non registered account\hrulefill \\
%\textbf{Vendor/Provider:} (Vendor name)\hrulefill \\
%\textbf{URL: } \url{http://www.google.de} \hrulefill \\
{\bf Categorization:} \\

\openup -1em
\vspace{-3em}

% Please, write either Multi-OS, Linux, Windows, Mac
\begin{tabular}{p{.5\textwidth}p{.5\textwidth}}

\begin{todolist}
  	\item Multi-OS
	\item Linux
\end{todolist}
&
\begin{todolist}
	\item[\done] Windows
	\item macOS
\end{todolist}

\end{tabular}

\openup 1em

{\bf SMO products in scope:} \\

\openup -1em
\vspace{-3em}

\begin{tabular}{p{.5\textwidth}p{.5\textwidth}}

\begin{todolist}
  \item[\done] Generic
  \item[\done] VICOS OC 100
  \item[\done] VICOS S\&D
\end{todolist}
&
\begin{todolist}
  \item Rail9000
  \item DCS
  \item Others: \hrulefill
\end{todolist}

\end{tabular}

\openup 1em

{\bf Description:} \\

\openup -1em
\vspace{-2em}

%\begin{itemize}
%	\item (What does the rule do?)
%	\item (Why was the situation targeted (e.g., in a workshop)?)
%\end{itemize}

This rule identifies situations in which a non-allowlisted user account successfully operated a Windows login. The rule takes advantage of Windows AppLocker, a Windows utility (introduced with Microsoft Windows 7 operating systems)  built to allow an administrator to restrict which programs users can execute based on several features (e.g., programs` paths, publishers, hashes, etc.) or with a custom Group Policy. Situations such as this one might indicate a discrepancy between the known accounts (the one allowlisted) and the actual accounts capable of logging in. In some cases, new accounts can be artificiously created (e.g., not through the available interface but directly added to the list of available users) to maintain access to the operating system.

\openup 1em

{\bf Rule query:} \\

\openup -1em
\vspace{-2em}

\begin{lstlisting}[language=json,firstnumber=1]
{
  "query": {
    "bool": {
      "must": [
        {
          "bool": {
            "should": [
              {
                "match": {
                  "eventId": "4624"
                }
              },
              {
                "match": {
                  "eventId": "528"
                }
              }
            ]
          }
        },
        {
          "bool": {
            "must_not": [
              {
                "terms": {
                  "objectUserName": [
                    "username1",
                    "username2",
                    "..."
                  ]
                }
              }
            ]
          }
        },
        {
          "match": {
            "cat": "winEvent"
          }
        },
        {
          "range": {
            "@timestamp": {
              "gt": "<bt>",
              "lt": "<et>"
            }
          }
        }
      ]
    }
  },
  "size": 1000
}
\end{lstlisting}

\openup 1em

{\bf Rule requirements:} \\

\openup -1em
\vspace{-2em}

Windows Applocker needs to be installed and running.

\openup 1em

{\bf Rule parameters:} \\

\openup -1em
\vspace{-2em}

\begin{itemize}
	\item A standard time frame defined by the ``greater than'' (gt) and ``less than'' parameters.
	\item \emph{usernameX}: A allowlisted username
\end{itemize}

\openup 1em

{\bf Further information/categorization:} \\

\openup -1em
\vspace{-2em}

%\begin{itemize}
%	\item <Does the rule have an ATT\&CK label?>
%	\item <Might the rule create many false positives?>
%\end{itemize}

The rule can be labeled with the following MITRE ATT\&CK's tactics and techniques:
\begin{itemize}
	\item T1078 - ``Valid Accounts''
		\subitem{.003 - `` Local Accounts''}
\end{itemize}

\pagebreak


% !TEX root = ../AdditionalOsaRules.tex

%%%
%
% Version 0.1
%
%%%

\section{Login Logout At Unusual Time}
\label{15022}
%\addcontentsline{toc}{section}{\protect\numberline{}(Tool)}%

\openup 1em

\textbf{Full name:} Login logout at unusual time\hrulefill \\
%\textbf{Vendor/Provider:} (Vendor name)\hrulefill \\
%\textbf{URL: } \url{http://www.google.de} \hrulefill \\
{\bf Categorization:} \\

\openup -1em
\vspace{-3em}

% Please, write either Multi-OS, Linux, Windows, Mac
\begin{tabular}{p{.5\textwidth}p{.5\textwidth}}

\begin{todolist}
  	\item Multi-OS
	\item Linux
\end{todolist}
&
\begin{todolist}
	\item[\done] Windows
	\item macOS
\end{todolist}

\end{tabular}

\openup 1em

{\bf SMO products in scope:} \\

\openup -1em
\vspace{-3em}

\begin{tabular}{p{.5\textwidth}p{.5\textwidth}}

\begin{todolist}
  \item[\done] Generic
  \item[\done] VICOS OC 100
  \item[\done] VICOS S\&D
\end{todolist}
&
\begin{todolist}
  \item Rail9000
  \item DCS
  \item Others: \hrulefill
\end{todolist}

\end{tabular}

\openup 1em

{\bf Description:} \\

\openup -1em
\vspace{-2em}

%\begin{itemize}
%	\item (What does the rule do?)
%	\item (Why was the situation targeted (e.g., in a workshop)?)
%\end{itemize}

This rule identifies situations in which Windows login and logout operations are performed or attempted outside the normal working hours (e.g., whenever operators work in shifts, this rule can ensure that such login and logout operations happen in the supposed timeframe). As attackers can try to take advantage of the absence of personnel to exploit the target systems, rules like these one work on restricting the categories of security-related events happening at unexpected time.

\openup 1em

{\bf Rule query:} \\

\openup -1em
\vspace{-2em}

\begin{lstlisting}[language=json,firstnumber=1]
{
  "runtime_mappings": {
    "daytime": {
      "type": "long",
      "script": {
        "source": "emit(doc['@timestamp'].value.toEpochSecond()%86400)"
      }
    }
  },
  "query": {
    "bool": {
      "must": [
        {
          "bool": {
            "should": [
              {
                "match": {
                  "eventId": "4624"
                }
              },
              {
                "match": {
                  "eventId": "4634"
                }
              },
              {
                "match": {
                  "eventId": "528"
                }
              },
              {
                "match": {
                  "eventId": "538"
                }
              }
            ]
          }
        },
        {
          "match": {
            "cat": "winEvent"
          }
        },
        {
          "bool": {
            "should": [
              {
                "range": {
                  "daytime": {
                    "lte": "shitfStartingTime"
                  }
                }
              },
              {
                "range": {
                  "daytime": {
                    "gte": "shitfFinishingTime"
                  }
                }
              },
              {
                "range": {
                  "@timestamp": {
                    "gt": "<bt>",
                    "lt": "<et>"
                  }
                }
              }
            ]
          }
        }
      ]
    }
  },
  "fields": [
    "daytime"
  ]
}
\end{lstlisting}

\openup 1em

{\bf Rule requirements:} \\

\openup -1em
\vspace{-2em}

None.

\openup 1em

{\bf Rule parameters:} \\

\openup -1em
\vspace{-2em}

\begin{itemize}
	\item A standard time frame defined by the ``greater than'' (gt) and ``less than'' parameters.
	\item \emph{shitfStartingTime}: The number of seconds from midnights indicating the normal shift starting time (e.g., 28800 would indicate ``8:00'')
	\item \emph{shitfFinishingTime}: The number of seconds from midnights indicating the normal shift ending time (e.g., 64800 would indicate ``18:00'')
\end{itemize}

\openup 1em

{\bf Further information/categorization:} \\

\openup -1em
\vspace{-2em}

%\begin{itemize}
%	\item <Does the rule have an ATT\&CK label?>
%	\item <Might the rule create many false positives?>
%\end{itemize}

The rule can be labeled with the following MITRE ATT\&CK's tactics and techniques:
\begin{itemize}
	\item T1078 - ``Valid Accounts''
		\subitem{.003 - `` Local Accounts''}
\end{itemize}

\pagebreak


\input{rules/15023_ApplicationInstallation}
\input{rules/15024_ApplicationRemoval}
% !TEX root = ../AdditionalOsaRules.tex

%%%
%
% Version 0.1
%
%%%

\section{Application Installation Applocker}
\label{15025}
%\addcontentsline{toc}{section}{\protect\numberline{}(Tool)}%

\openup 1em

\textbf{Full name:} Application installation applocker\hrulefill \\
%\textbf{Vendor/Provider:} (Vendor name)\hrulefill \\
%\textbf{URL: } \url{http://www.google.de} \hrulefill \\
{\bf Categorization:} \\

\openup -1em
\vspace{-3em}

% Please, write either Multi-OS, Linux, Windows, Mac
\begin{tabular}{p{.5\textwidth}p{.5\textwidth}}

\begin{todolist}
  	\item Multi-OS
	\item Linux
\end{todolist}
&
\begin{todolist}
	\item[\done] Windows
	\item macOS
\end{todolist}

\end{tabular}

\openup 1em

{\bf SMO products in scope:} \\

\openup -1em
\vspace{-3em}

\begin{tabular}{p{.5\textwidth}p{.5\textwidth}}

\begin{todolist}
  \item[\done] Generic
  \item VICOS OC 100
  \item VICOS S\&D
\end{todolist}
&
\begin{todolist}
  \item Rail9000
  \item DCS
  \item Others: \hrulefill
\end{todolist}

\end{tabular}

\openup 1em

{\bf Description:} \\

\openup -1em
\vspace{-2em}

%\begin{itemize}
%	\item (What does the rule do?)
%	\item (Why was the situation targeted (e.g., in a workshop)?)
%\end{itemize}

This rule identifies situations in which a new Windows application is installed in Windows. Windows AppLocker (introduced with Microsoft Windows 7 operating systems)  is an application that allows an administrator to restrict which programs users can execute based on several features (e.g., programs` paths, publishers, hashes, etc.) or with a custom Group Policy.Differently from rule 15023 ("Application Installation"), this rule takes advantage of Windows utility Applocker to monitor which applications are installed in the system. An alert raised by this rule will indicate that the specific application was also previously allowlisted by Applocker. As for rule 15023, attackers can create and consequently run new applications on a system to achieve several objectives (e.g., ensure persistence, damage the system, etc.).

\openup 1em

{\bf Rule query:} \\

\openup -1em
\vspace{-2em}

\begin{lstlisting}[language=json,firstnumber=1]
{
  "query": {
    "bool": {
      "must": [
        {
          "match": {
            "eventId": "8030"
          }
        },
        {
          "match": {
            "cat": "winEvent"
          }
        },
        {
          "range": {
            "@timestamp": {
              "gt": "<bt>",
              "lt": "<et>"
            }
          }
        }
      ]
    }
  },
  "size": 1000
}
\end{lstlisting}

\openup 1em

{\bf Rule requirements:} \\

\openup -1em
\vspace{-2em}

Windows Applocker needs to be installed and running.

\openup 1em

{\bf Rule parameters:} \\

\openup -1em
\vspace{-2em}

\begin{itemize}
	\item A standard time frame defined by the ``greater than'' (gt) and ``less than'' parameters.
\end{itemize}

\openup 1em

{\bf Further information/categorization:} \\

\openup -1em
\vspace{-2em}

%\begin{itemize}
%	\item <Does the rule have an ATT\&CK label?>
%	\item <Might the rule create many false positives?>
%\end{itemize}

The rule can be labeled with the following MITRE ATT\&CK's tactics and techniques:
\begin{itemize}
	\item T1543 - ``Create or Modify System Process''
\end{itemize}

\pagebreak


% !TEX root = ../AdditionalOsaRules.tex

%%%
%
% Version 0.1
%
%%%

\section{Os High Privileged Account Creation}
\label{15026}
%\addcontentsline{toc}{section}{\protect\numberline{}(Tool)}%

\openup 1em

\textbf{Full name:} Os high privileged account creation\hrulefill \\
%\textbf{Vendor/Provider:} (Vendor name)\hrulefill \\
%\textbf{URL: } \url{http://www.google.de} \hrulefill \\
{\bf Categorization:} \\

\openup -1em
\vspace{-3em}

% Please, write either Multi-OS, Linux, Windows, Mac
\begin{tabular}{p{.5\textwidth}p{.5\textwidth}}

\begin{todolist}
  	\item Multi-OS
	\item Linux
\end{todolist}
&
\begin{todolist}
	\item[\done] Windows
	\item macOS
\end{todolist}

\end{tabular}

\openup 1em

{\bf SMO products in scope:} \\

\openup -1em
\vspace{-3em}

\begin{tabular}{p{.5\textwidth}p{.5\textwidth}}

\begin{todolist}
  \item[\done] Generic
  \item[\done] VICOS OC 100
  \item[\done] VICOS S\&D
\end{todolist}
&
\begin{todolist}
  \item Rail9000
  \item DCS
  \item Others: \hrulefill
\end{todolist}

\end{tabular}

\openup 1em

{\bf Description:} \\

\openup -1em
\vspace{-2em}

%\begin{itemize}
%	\item (What does the rule do?)
%	\item (Why was the situation targeted (e.g., in a workshop)?)
%\end{itemize}

This rule is used to identify situations in which an attacker creates a new Windows account with high privileges. Actions such as this one might indicate that a successful privilege escalation has occurred and the attacker can ensure persistence on the machine by creating new resources. Rules related to new accounts being created can be enhanced in a number of ways. Among the most common: ensuring that the ``SAM Account Name' and the ``User Account Name' are not empty, ensuring that ``Home Directory' abnd ``Home Drive' are empty (typical for new user accounts), and checking that the ``Primary Group ID' is equal to 513 (typical for domain and local users).

\openup 1em

{\bf Rule query:} \\

\openup -1em
\vspace{-2em}

\begin{lstlisting}[language=json,firstnumber=1]
{
  "query": {
    "bool": {
      "must": [
        {
          "bool": {
            "should": [
              {
                "match": {
                  "eventId": "4720"
                }
              },
              {
                "match": {
                  "eventId": "624"
                }
              }
            ]
          }
        },
        {
          "bool": {
            "must_not": [
              {
                "match_phrase": {
                  "event_data.UserAccountControl": "%%2080 %%2082 %%2084"
                }
              }
            ]
          }
        },
        {
          "match": {
            "cat": "winEvent"
          }
        },
        {
          "range": {
            "@timestamp": {
              "gt": "<bt>",
              "lt": "<et>"
            }
          }
        }
      ]
    }
  },
  "size": 1000
}
\end{lstlisting}

\openup 1em

{\bf Rule requirements:} \\

\openup -1em
\vspace{-2em}

None.

\openup 1em

{\bf Rule parameters:} \\

\openup -1em
\vspace{-2em}

\begin{itemize}
	\item A standard time frame defined by the ``greater than'' (gt) and ``less than'' parameters.
\end{itemize}

\openup 1em

{\bf Further information/categorization:} \\

\openup -1em
\vspace{-2em}

%\begin{itemize}
%	\item <Does the rule have an ATT\&CK label?>
%	\item <Might the rule create many false positives?>
%\end{itemize}

The rule can be labeled with the following MITRE ATT\&CK's tactics and techniques:
\begin{itemize}
	\item T1136 - ``Create Account''
		\subitem{.001 - `` Local Account''}
	\item T1136 - ``Create Account''
		\subitem{.002 - `` Domain Account''}
\end{itemize}

\pagebreak


\input{rules/15027_ApplicationConfigurationChange}
% !TEX root = ../AdditionalOsaRules.tex

%%%
%
% Version 0.1
%
%%%

\section{Windows Object Deletion}
\label{15028}
%\addcontentsline{toc}{section}{\protect\numberline{}(Tool)}%

\openup 1em

\textbf{Full name:} Windows object deletion\hrulefill \\
%\textbf{Vendor/Provider:} (Vendor name)\hrulefill \\
%\textbf{URL: } \url{http://www.google.de} \hrulefill \\
{\bf Categorization:} \\

\openup -1em
\vspace{-3em}

% Please, write either Multi-OS, Linux, Windows, Mac
\begin{tabular}{p{.5\textwidth}p{.5\textwidth}}

\begin{todolist}
  	\item Multi-OS
	\item Linux
\end{todolist}
&
\begin{todolist}
	\item[\done] Windows
	\item macOS
\end{todolist}

\end{tabular}

\openup 1em

{\bf SMO products in scope:} \\

\openup -1em
\vspace{-3em}

\begin{tabular}{p{.5\textwidth}p{.5\textwidth}}

\begin{todolist}
  \item[\done] Generic
  \item[\done] VICOS OC 100
  \item VICOS S\&D
\end{todolist}
&
\begin{todolist}
  \item Rail9000
  \item DCS
  \item Others: \hrulefill
\end{todolist}

\end{tabular}

\openup 1em

{\bf Description:} \\

\openup -1em
\vspace{-2em}

%\begin{itemize}
%	\item (What does the rule do?)
%	\item (Why was the situation targeted (e.g., in a workshop)?)
%\end{itemize}

This rule identifies situations in which specific resources in Windows have been selected for deletion. Rules such as this one might help operators to monitor valuable resources (e.g., log files) and avoid attackers to make them unavailable for tracking, forensics, recovery, etc.

\openup 1em

{\bf Rule query:} \\

\openup -1em
\vspace{-2em}

\begin{lstlisting}[language=json,firstnumber=1]
{
  "query": {
    "bool": {
      "must": [
        {
          "match": {
            "eventId": "4659"
          }
        },
        {
          "match_phrase": {
            "s_targetFilename": "ResourceName*"
          }
        },
        {
          "match": {
            "cat": "winEvent"
          }
        },
        {
          "range": {
            "@timestamp": {
              "gt": "<bt>",
              "lt": "<et>"
            }
          }
        }
      ]
    }
  },
  "size": 1000
}
\end{lstlisting}

\openup 1em

{\bf Rule requirements:} \\

\openup -1em
\vspace{-2em}

None.

\openup 1em

{\bf Rule parameters:} \\

\openup -1em
\vspace{-2em}

\begin{itemize}
	\item A standard time frame defined by the ``greater than'' (gt) and ``less than'' parameters.
	\item \emph{ResourceName}: The resource (e.g., file, directory) to monitor
\end{itemize}

\openup 1em

{\bf Further information/categorization:} \\

\openup -1em
\vspace{-2em}

%\begin{itemize}
%	\item <Does the rule have an ATT\&CK label?>
%	\item <Might the rule create many false positives?>
%\end{itemize}

The rule can be labeled with the following MITRE ATT\&CK's tactics and techniques:
\begin{itemize}
	\item T1485 - ``Data Destruction''
\end{itemize}

\pagebreak



\input{rules/15030_NewHardwarePluggedIn}
% !TEX root = ../AdditionalOsaRules.tex

%%%
%
% Version 0.1
%
%%%

\section{New Usb Storage Device Plugged In}
\label{15031}
%\addcontentsline{toc}{section}{\protect\numberline{}(Tool)}%

\openup 1em

\textbf{Full name:} New usb storage device plugged in\hrulefill \\
%\textbf{Vendor/Provider:} (Vendor name)\hrulefill \\
%\textbf{URL: } \url{http://www.google.de} \hrulefill \\
{\bf Categorization:} \\

\openup -1em
\vspace{-3em}

% Please, write either Multi-OS, Linux, Windows, Mac
\begin{tabular}{p{.5\textwidth}p{.5\textwidth}}

\begin{todolist}
  	\item Multi-OS
	\item Linux
\end{todolist}
&
\begin{todolist}
	\item[\done] Windows
	\item macOS
\end{todolist}

\end{tabular}

\openup 1em

{\bf SMO products in scope:} \\

\openup -1em
\vspace{-3em}

\begin{tabular}{p{.5\textwidth}p{.5\textwidth}}

\begin{todolist}
  \item[\done] Generic
  \item[\done] VICOS OC 100
  \item[\done] VICOS S\&D
\end{todolist}
&
\begin{todolist}
  \item Rail9000
  \item DCS
  \item Others: \hrulefill
\end{todolist}

\end{tabular}

\openup 1em

{\bf Description:} \\

\openup -1em
\vspace{-2em}

%\begin{itemize}
%	\item (What does the rule do?)
%	\item (Why was the situation targeted (e.g., in a workshop)?)
%\end{itemize}

This rules triggers every time a new external usb storage device is connected to and recognized by a Windows operating system. Attackers can normally use this kind of devices to access a system (e.g., by delivering a malware) or extract information from the system. The rule can be further specified by defining the kind of devices that are allowed (or not allowed) by looking at properties such as ``Device ID'', ``Device Name'', ``Vendor ID'', etc as well as ensuring that the ``Subject\textbackslash Security ID'' is set to ``SYSTEM''.

\openup 1em

{\bf Rule query:} \\

\openup -1em
\vspace{-2em}

\begin{lstlisting}[language=json,firstnumber=1]
{
  "query": {
    "bool": {
      "must": [
        {
          "match": {
            "cat": "winEvent"
          }
        },
        {
          "bool": {
            "should": [
              {
                "match": {
                  "eventId": "2100"
                }
              },
              {
                "match": {
                  "eventId": "2101"
                }
              }
            ]
          }
        },
        {
          "match_phrase_prefix": {
            "msg": "*_USBSTOR*"
          }
        },
        {
          "range": {
            "@timestamp": {
              "gt": "<bt>",
              "lt": "<et>"
            }
          }
        }
      ]
    }
  },
  "size": 1000
}
\end{lstlisting}

\openup 1em

{\bf Rule requirements:} \\

\openup -1em
\vspace{-2em}

None.

\openup 1em

{\bf Rule parameters:} \\

\openup -1em
\vspace{-2em}

\begin{itemize}
	\item A standard time frame defined by the ``greater than'' (gt) and ``less than'' parameters.
\end{itemize}

\openup 1em

{\bf Further information/categorization:} \\

\openup -1em
\vspace{-2em}

%\begin{itemize}
%	\item <Does the rule have an ATT\&CK label?>
%	\item <Might the rule create many false positives?>
%\end{itemize}

The rule can be labeled with the following MITRE ATT\&CK's tactics and techniques:
\begin{itemize}
	\item T1200 - ``Hardware Additions''
	\item T1025 - ``Data from Removable Media''
\end{itemize}

\pagebreak


% !TEX root = ../AdditionalOsaRules.tex

%%%
%
% Version 0.1
%
%%%

\section{New Usb Device Plugged In}
\label{15032}
%\addcontentsline{toc}{section}{\protect\numberline{}(Tool)}%

\openup 1em

\textbf{Full name:} New usb device plugged in\hrulefill \\
%\textbf{Vendor/Provider:} (Vendor name)\hrulefill \\
%\textbf{URL: } \url{http://www.google.de} \hrulefill \\
{\bf Categorization:} \\

\openup -1em
\vspace{-3em}

% Please, write either Multi-OS, Linux, Windows, Mac
\begin{tabular}{p{.5\textwidth}p{.5\textwidth}}

\begin{todolist}
  	\item Multi-OS
	\item Linux
\end{todolist}
&
\begin{todolist}
	\item[\done] Windows
	\item macOS
\end{todolist}

\end{tabular}

\openup 1em

{\bf SMO products in scope:} \\

\openup -1em
\vspace{-3em}

\begin{tabular}{p{.5\textwidth}p{.5\textwidth}}

\begin{todolist}
  \item[\done] Generic
  \item[\done] VICOS OC 100
  \item[\done] VICOS S\&D
\end{todolist}
&
\begin{todolist}
  \item Rail9000
  \item DCS
  \item Others: \hrulefill
\end{todolist}

\end{tabular}

\openup 1em

{\bf Description:} \\

\openup -1em
\vspace{-2em}

%\begin{itemize}
%	\item (What does the rule do?)
%	\item (Why was the situation targeted (e.g., in a workshop)?)
%\end{itemize}

This rules triggers every time a new external usb device is connected to and recognized by a Windows operating system. This rule generalize rule 15031 ("New Usb Storage Device Plugged-in") and, as of that rule, it can be specified by defining the kind of devices that are allowed (or not allowed) by looking at properties such as ``Device ID', ``Device Name', ``Vendor ID', etc as well as ensuring that the ``Subject\\textbackslash Security ID' is set to ``SYSTEM'.

\openup 1em

{\bf Rule query:} \\

\openup -1em
\vspace{-2em}

\begin{lstlisting}[language=json,firstnumber=1]
{
  "query": {
    "bool": {
      "must": [
        {
          "match": {
            "cat": "winEvent"
          }
        },
        {
          "bool": {
            "should": [
              {
                "match": {
                  "eventId": "2100"
                }
              },
              {
                "match": {
                  "eventId": "2101"
                }
              }
            ]
          }
        },
        {
          "range": {
            "@timestamp": {
              "gt": "<bt>",
              "lt": "<et>"
            }
          }
        }
      ]
    }
  },
  "size": 1000
}
\end{lstlisting}

\openup 1em

{\bf Rule requirements:} \\

\openup -1em
\vspace{-2em}

None.

\openup 1em

{\bf Rule parameters:} \\

\openup -1em
\vspace{-2em}

\begin{itemize}
	\item A standard time frame defined by the ``greater than'' (gt) and ``less than'' parameters.
\end{itemize}

\openup 1em

{\bf Further information/categorization:} \\

\openup -1em
\vspace{-2em}

%\begin{itemize}
%	\item <Does the rule have an ATT\&CK label?>
%	\item <Might the rule create many false positives?>
%\end{itemize}

The rule can be labeled with the following MITRE ATT\&CK's tactics and techniques:
\begin{itemize}
	\item T1200 - ``Hardware Additions''
\end{itemize}

\pagebreak



% !TEX root = ../AdditionalOsaRules.tex

%%%
%
% Version 0.1
%
%%%

\section{New File Share}
\label{15041}
%\addcontentsline{toc}{section}{\protect\numberline{}(Tool)}%

\openup 1em

\textbf{Full name:} New file share\hrulefill \\
%\textbf{Vendor/Provider:} (Vendor name)\hrulefill \\
%\textbf{URL: } \url{http://www.google.de} \hrulefill \\
{\bf Categorization:} \\

\openup -1em
\vspace{-3em}

% Please, write either Multi-OS, Linux, Windows, Mac
\begin{tabular}{p{.5\textwidth}p{.5\textwidth}}

\begin{todolist}
  	\item Multi-OS
	\item Linux
\end{todolist}
&
\begin{todolist}
	\item[\done] Windows
	\item macOS
\end{todolist}

\end{tabular}

\openup 1em

{\bf SMO products in scope:} \\

\openup -1em
\vspace{-3em}

\begin{tabular}{p{.5\textwidth}p{.5\textwidth}}

\begin{todolist}
  \item[\done] Generic
  \item[\done] VICOS OC 100
  \item[\done] VICOS S\&D
\end{todolist}
&
\begin{todolist}
  \item Rail9000
  \item DCS
  \item Others: \hrulefill
\end{todolist}

\end{tabular}

\openup 1em

{\bf Description:} \\

\openup -1em
\vspace{-2em}

%\begin{itemize}
%	\item (What does the rule do?)
%	\item (Why was the situation targeted (e.g., in a workshop)?)
%\end{itemize}

This rule identifies situations in which a network share object is created. File shares can be used by attackers to spread across systems sharing a network. Depending on the context, this rule can be further specified by checking the actual ``Share Name'' of the accessed share folder (to be used in case specific shared folder are supposed to be created during normal operations) or by monitoring if the ``Network Information \\ Source Address'' is not from an internal IP range.

\openup 1em

{\bf Rule query:} \\

\openup -1em
\vspace{-2em}

\begin{lstlisting}[language=json,firstnumber=1]
{
  "query": {
    "bool": {
      "must": [
        {
          "match": {
            "eventId": "5142"
          }
        },
        {
          "match": {
            "cat": "winEvent"
          }
        },
        {
          "range": {
            "@timestamp": {
              "gt": "<bt>",
              "lt": "<et>"
            }
          }
        }
      ]
    }
  },
  "size": 1000
}
\end{lstlisting}

\openup 1em

{\bf Rule requirements:} \\

\openup -1em
\vspace{-2em}

None.

\openup 1em

{\bf Rule parameters:} \\

\openup -1em
\vspace{-2em}

\begin{itemize}
	\item A standard time frame defined by the ``greater than'' (gt) and ``less than'' parameters.
\end{itemize}

\openup 1em

{\bf Further information/categorization:} \\

\openup -1em
\vspace{-2em}

%\begin{itemize}
%	\item <Does the rule have an ATT\&CK label?>
%	\item <Might the rule create many false positives?>
%\end{itemize}

None.

\pagebreak



\input{rules/15050_NewFileOrFolder}
% !TEX root = ../AdditionalOsaRules.tex

%%%
%
% Version 0.1
%
%%%

\section{Deleted File Or Folder}
\label{15051}
%\addcontentsline{toc}{section}{\protect\numberline{}(Tool)}%

\openup 1em

\textbf{Full name:} Deleted file or folder\hrulefill \\
%\textbf{Vendor/Provider:} (Vendor name)\hrulefill \\
%\textbf{URL: } \url{http://www.google.de} \hrulefill \\
{\bf Categorization:} \\

\openup -1em
\vspace{-3em}

% Please, write either Multi-OS, Linux, Windows, Mac
\begin{tabular}{p{.5\textwidth}p{.5\textwidth}}

\begin{todolist}
  	\item Multi-OS
	\item Linux
\end{todolist}
&
\begin{todolist}
	\item[\done] Windows
	\item macOS
\end{todolist}

\end{tabular}

\openup 1em

{\bf SMO products in scope:} \\

\openup -1em
\vspace{-3em}

\begin{tabular}{p{.5\textwidth}p{.5\textwidth}}

\begin{todolist}
  \item[\done] Generic
  \item[\done] VICOS OC 100
  \item[\done] VICOS S\&D
\end{todolist}
&
\begin{todolist}
  \item Rail9000
  \item DCS
  \item Others: \hrulefill
\end{todolist}

\end{tabular}

\openup 1em

{\bf Description:} \\

\openup -1em
\vspace{-2em}

%\begin{itemize}
%	\item (What does the rule do?)
%	\item (Why was the situation targeted (e.g., in a workshop)?)
%\end{itemize}

This rule identifies situations in which a known (and likely used) file or folder is deleted from a Windows system. The unexpected deletion of files or folders might be due to attackers' activities on the system and can have different meanings (e.g., moving data, compromise system's functions, etc.). Such a rule is very generic and shall be normally used in quite static environments and always used with care no to raise false positives. Alternatively, the rule can be further specified (e.g., restricted) by using field "s\_targetFilename" to identify a given section of the file system.

\openup 1em

{\bf Rule query:} \\

\openup -1em
\vspace{-2em}

\begin{lstlisting}[language=json,firstnumber=1]
{
  "query": {
    "bool": {
      "must": [
        {
          "bool": {
            "should": [
              {
                "match": {
                  "eventId": "23"
                }
              },
              {
                "match": {
                  "eventId": "26"
                }
              }
            ]
          }
        },
        {
          "match": {
            "cat": "winEvent"
          }
        },
        {
          "range": {
            "@timestamp": {
              "gt": "<bt>",
              "lt": "<et>"
            }
          }
        }
      ]
    }
  },
  "size": 1000
}
\end{lstlisting}

\openup 1em

{\bf Rule requirements:} \\

\openup -1em
\vspace{-2em}

This rule requires the installation of the Windows Windows Sysmon service.

\openup 1em

{\bf Rule parameters:} \\

\openup -1em
\vspace{-2em}

\begin{itemize}
	\item A standard time frame defined by the ``greater than'' (gt) and ``less than'' parameters.
\end{itemize}

\openup 1em

{\bf Further information/categorization:} \\

\openup -1em
\vspace{-2em}

%\begin{itemize}
%	\item <Does the rule have an ATT\&CK label?>
%	\item <Might the rule create many false positives?>
%\end{itemize}

None.

\pagebreak


\input{rules/15052_AccessedFileOrFolder}
% !TEX root = ../AdditionalOsaRules.tex

%%%
%
% Version 0.1
%
%%%

\section{Modified File Or Folder}
\label{15053}
%\addcontentsline{toc}{section}{\protect\numberline{}(Tool)}%

\openup 1em

\textbf{Full name:} Modified file or folder\hrulefill \\
%\textbf{Vendor/Provider:} (Vendor name)\hrulefill \\
%\textbf{URL: } \url{http://www.google.de} \hrulefill \\
{\bf Categorization:} \\

\openup -1em
\vspace{-3em}

% Please, write either Multi-OS, Linux, Windows, Mac
\begin{tabular}{p{.5\textwidth}p{.5\textwidth}}

\begin{todolist}
  	\item Multi-OS
	\item Linux
\end{todolist}
&
\begin{todolist}
	\item[\done] Windows
	\item macOS
\end{todolist}

\end{tabular}

\openup 1em

{\bf SMO products in scope:} \\

\openup -1em
\vspace{-3em}

\begin{tabular}{p{.5\textwidth}p{.5\textwidth}}

\begin{todolist}
  \item[\done] Generic
  \item[\done] VICOS OC 100
  \item[\done] VICOS S\&D
\end{todolist}
&
\begin{todolist}
  \item Rail9000
  \item DCS
  \item Others: \hrulefill
\end{todolist}

\end{tabular}

\openup 1em

{\bf Description:} \\

\openup -1em
\vspace{-2em}

%\begin{itemize}
%	\item (What does the rule do?)
%	\item (Why was the situation targeted (e.g., in a workshop)?)
%\end{itemize}

This rule identifies situations in which a known file or folder is modified in a Windows system. The unexpected modification of files or folders might be due to attackers' activities on the system and can have different meanings (e.g., data tampering, etc.). Such a rule is very generic and shall be normally used in quite static environments and always used with care no to raise false positives. Alternatively, the rule can be further specified (e.g., restricted) by using field "s\_targetFilename" to identify a given section of the file system.

\openup 1em

{\bf Rule query:} \\

\openup -1em
\vspace{-2em}

\begin{lstlisting}[language=json,firstnumber=1]
{
  "query": {
    "bool": {
      "must": [
        {
          "match": {
            "eventId": "4663"
          }
        },
        {
          "terms": {
            "event_data.AccessList": [
              "%%4417",
              "%%4418",
              "%%4420",
              "%%4424",
              "%%1539",
              "%%1540"
            ]
          }
        },
        {
          "match": {
            "cat": "winEvent"
          }
        },
        {
          "range": {
            "@timestamp": {
              "gt": "<bt>",
              "lt": "<et>"
            }
          }
        }
      ]
    }
  },
  "size": 1000
}
\end{lstlisting}

\openup 1em

{\bf Rule requirements:} \\

\openup -1em
\vspace{-2em}

This rule requires the directory/file to be configured so that the event is generated (for more information, please, visit the following link https://www.manageengine.com/products/active-directory-audit/how-to/monitor-file-and-folder-access-on-windows-file-server.html)

\openup 1em

{\bf Rule parameters:} \\

\openup -1em
\vspace{-2em}

\begin{itemize}
	\item A standard time frame defined by the ``greater than'' (gt) and ``less than'' parameters.
\end{itemize}

\openup 1em

{\bf Further information/categorization:} \\

\openup -1em
\vspace{-2em}

%\begin{itemize}
%	\item <Does the rule have an ATT\&CK label?>
%	\item <Might the rule create many false positives?>
%\end{itemize}

None.

\pagebreak



\input{rules/15060_MaliciousProcessExecution}
% !TEX root = ../AdditionalOsaRules.tex

%%%
%
% Version 0.1
%
%%%

\section{Anomalous Process Execution}
\label{15061}
%\addcontentsline{toc}{section}{\protect\numberline{}(Tool)}%

\openup 1em

\textbf{Full name:} Anomalous process execution\hrulefill \\
%\textbf{Vendor/Provider:} (Vendor name)\hrulefill \\
%\textbf{URL: } \url{http://www.google.de} \hrulefill \\
{\bf Categorization:} \\

\openup -1em
\vspace{-3em}

% Please, write either Multi-OS, Linux, Windows, Mac
\begin{tabular}{p{.5\textwidth}p{.5\textwidth}}

\begin{todolist}
  	\item Multi-OS
	\item Linux
\end{todolist}
&
\begin{todolist}
	\item[\done] Windows
	\item macOS
\end{todolist}

\end{tabular}

\openup 1em

{\bf SMO products in scope:} \\

\openup -1em
\vspace{-3em}

\begin{tabular}{p{.5\textwidth}p{.5\textwidth}}

\begin{todolist}
  \item[\done] Generic
  \item[\done] VICOS OC 100
  \item[\done] VICOS S\&D
\end{todolist}
&
\begin{todolist}
  \item Rail9000
  \item DCS
  \item Others: \hrulefill
\end{todolist}

\end{tabular}

\openup 1em

{\bf Description:} \\

\openup -1em
\vspace{-2em}

%\begin{itemize}
%	\item (What does the rule do?)
%	\item (Why was the situation targeted (e.g., in a workshop)?)
%\end{itemize}

This rule identifies situations in which a Windows system executes an unknown process. This event might occur especially in the initial phases of an attack (e.g., attacker running a given malware toolkits to acquire privileges) but it can generally happen in any phase of a cyber intrusion. Differently from rule 15060 ("Malicious Process Execution"), this rule should be used only in stable system where the list of running processes can be described with reasonable certainty (not to rise too many false positives).

\openup 1em

{\bf Rule query:} \\

\openup -1em
\vspace{-2em}

\begin{lstlisting}[language=json,firstnumber=1]
{
  "query": {
    "bool": {
      "must": [
        {
          "match": {
            "eventId": "1"
          }
        },
        {
          "bool": {
            "must_not": [
              {
                "terms": {
                  "event_data.NewProcessName": [
                    "executable1",
                    "executable2",
                    "..."
                  ]
                }
              }
            ]
          }
        },
        {
          "match": {
            "cat": "winEvent"
          }
        },
       {
          "range": {
            "@timestamp": {
              "gt": "<bt>",
              "lt": "<et>"
            }
          }
        }
      ]
    }
  },
  "size": 1000
}
\end{lstlisting}

\openup 1em

{\bf Rule requirements:} \\

\openup -1em
\vspace{-2em}

None.

\openup 1em

{\bf Rule parameters:} \\

\openup -1em
\vspace{-2em}

\begin{itemize}
	\item A standard time frame defined by the ``greater than'' (gt) and ``less than'' parameters.
	\item \emph{executableX}: A allowlisted process name
\end{itemize}

\openup 1em

{\bf Further information/categorization:} \\

\openup -1em
\vspace{-2em}

%\begin{itemize}
%	\item <Does the rule have an ATT\&CK label?>
%	\item <Might the rule create many false positives?>
%\end{itemize}

The rule can be labeled with the following MITRE ATT\&CK's tactics and techniques:
\begin{itemize}
	\item T1204 - ``User Execution''
		\subitem{.002 - `` Malicious File''}
\end{itemize}

\pagebreak


% !TEX root = ../AdditionalOsaRules.tex

%%%
%
% Version 0.1
%
%%%

\section{Anomalous Process Execution With High Privileges}
\label{15062}
%\addcontentsline{toc}{section}{\protect\numberline{}(Tool)}%

\openup 1em

\textbf{Full name:} Anomalous process execution with high privileges\hrulefill \\
%\textbf{Vendor/Provider:} (Vendor name)\hrulefill \\
%\textbf{URL: } \url{http://www.google.de} \hrulefill \\
{\bf Categorization:} \\

\openup -1em
\vspace{-3em}

% Please, write either Multi-OS, Linux, Windows, Mac
\begin{tabular}{p{.5\textwidth}p{.5\textwidth}}

\begin{todolist}
  	\item Multi-OS
	\item Linux
\end{todolist}
&
\begin{todolist}
	\item[\done] Windows
	\item macOS
\end{todolist}

\end{tabular}

\openup 1em

{\bf SMO products in scope:} \\

\openup -1em
\vspace{-3em}

\begin{tabular}{p{.5\textwidth}p{.5\textwidth}}

\begin{todolist}
  \item[\done] Generic
  \item[\done] VICOS OC 100
  \item[\done] VICOS S\&D
\end{todolist}
&
\begin{todolist}
  \item Rail9000
  \item DCS
  \item Others: \hrulefill
\end{todolist}

\end{tabular}

\openup 1em

{\bf Description:} \\

\openup -1em
\vspace{-2em}

%\begin{itemize}
%	\item (What does the rule do?)
%	\item (Why was the situation targeted (e.g., in a workshop)?)
%\end{itemize}

This rule identifies situations in which a Windows system executes an unknown process. This event might occur especially in the initial phases of an attack (e.g., attacker running a given malware toolkits to acquire privileges) but it can generally happen in any phase of a cyber intrusion. Compared rule 15061 ("Anomalous Process Execution"), this rule narrows the scope to highly-privileged processes (e.g., processes that have the right to change operating system configurations and specifications).

\openup 1em

{\bf Rule query:} \\

\openup -1em
\vspace{-2em}

\begin{lstlisting}[language=json,firstnumber=1]
{
  "query": {
    "bool": {
      "must": [
        {
          "match": {
            "eventId": "1"
          }
        },
        {
          "bool": {
            "must_not": [
              {
                "terms": {
                  "event_data.NewProcessName": [
                    "executable1",
                    "executable2",
                    "..."
                  ]
                }
              }
            ]
          }
        },
        {
          "bool": {
            "should": [
              {
                "match": {
                  "event_data.TokenElevationType": "%%1937"
                }
              },
              {
                "match": {
                  "event_data.TokenElevationType": "%%1938"
                }
              }
            ]
          }
        },
        {
          "match": {
            "cat": "winEvent"
          }
        },
        {
          "range": {
            "@timestamp": {
              "gt": "<bt>",
              "lt": "<et>"
            }
          }
        }
      ]
    }
  },
  "size": 1000
}
\end{lstlisting}

\openup 1em

{\bf Rule requirements:} \\

\openup -1em
\vspace{-2em}

None.

\openup 1em

{\bf Rule parameters:} \\

\openup -1em
\vspace{-2em}

\begin{itemize}
	\item A standard time frame defined by the ``greater than'' (gt) and ``less than'' parameters.
	\item \emph{executableX}: A allowlisted process name
\end{itemize}

\openup 1em

{\bf Further information/categorization:} \\

\openup -1em
\vspace{-2em}

%\begin{itemize}
%	\item <Does the rule have an ATT\&CK label?>
%	\item <Might the rule create many false positives?>
%\end{itemize}

The rule can be labeled with the following MITRE ATT\&CK's tactics and techniques:
\begin{itemize}
	\item T1204 - ``User Execution''
		\subitem{.002 - `` Malicious File''}
\end{itemize}

\pagebreak


% !TEX root = ../AdditionalOsaRules.tex

%%%
%
% Version 0.1
%
%%%

\section{Process Termination}
\label{15063}
%\addcontentsline{toc}{section}{\protect\numberline{}(Tool)}%

\openup 1em

\textbf{Full name:} Process termination\hrulefill \\
%\textbf{Vendor/Provider:} (Vendor name)\hrulefill \\
%\textbf{URL: } \url{http://www.google.de} \hrulefill \\
{\bf Categorization:} \\

\openup -1em
\vspace{-3em}

% Please, write either Multi-OS, Linux, Windows, Mac
\begin{tabular}{p{.5\textwidth}p{.5\textwidth}}

\begin{todolist}
  	\item Multi-OS
	\item Linux
\end{todolist}
&
\begin{todolist}
	\item[\done] Windows
	\item macOS
\end{todolist}

\end{tabular}

\openup 1em

{\bf SMO products in scope:} \\

\openup -1em
\vspace{-3em}

\begin{tabular}{p{.5\textwidth}p{.5\textwidth}}

\begin{todolist}
  \item[\done] Generic
  \item[\done] VICOS OC 100
  \item[\done] VICOS S\&D
\end{todolist}
&
\begin{todolist}
  \item Rail9000
  \item DCS
  \item Others: \hrulefill
\end{todolist}

\end{tabular}

\openup 1em

{\bf Description:} \\

\openup -1em
\vspace{-2em}

%\begin{itemize}
%	\item (What does the rule do?)
%	\item (Why was the situation targeted (e.g., in a workshop)?)
%\end{itemize}

This rule identifies situations in which a known process running in a Windows system terminates. This event might occur if an attacker tries to subvert the correct functioning of the system or attempts to impair defenses (e.g., interrupting security tools running in the system). As for other rules concerning process executions (e.g., 15061 "Anomalous Process Execution"), this rule can be further specified by including information about the process name or folder (e.g., specifing a list of critical processes that should be running all the time) narrowing its scope.

\openup 1em

{\bf Rule query:} \\

\openup -1em
\vspace{-2em}

\begin{lstlisting}[language=json,firstnumber=1]
{
  "query": {
    "bool": {
      "must": [
        {
          "match": {
            "eventId": "4689"
          }
        },
        {
          "match": {
            "cat": "winEvent"
          }
        },
        {
          "range": {
            "@timestamp": {
              "gt": "<bt>",
              "lt": "<et>"
            }
          }
        }
      ]
    }
  },
  "size": 1000
}
\end{lstlisting}

\openup 1em

{\bf Rule requirements:} \\

\openup -1em
\vspace{-2em}

None.

\openup 1em

{\bf Rule parameters:} \\

\openup -1em
\vspace{-2em}

\begin{itemize}
	\item A standard time frame defined by the ``greater than'' (gt) and ``less than'' parameters.
	\item \emph{executableX}: A allowlisted process name
\end{itemize}

\openup 1em

{\bf Further information/categorization:} \\

\openup -1em
\vspace{-2em}

%\begin{itemize}
%	\item <Does the rule have an ATT\&CK label?>
%	\item <Might the rule create many false positives?>
%\end{itemize}

The rule can be labeled with the following MITRE ATT\&CK's tactics and techniques:
\begin{itemize}
	\item T1489 - ``Service Stop''
\end{itemize}

\pagebreak


\input{rules/15064_NewServiceExecution}

% !TEX root = ../AdditionalOsaRules.tex

%%%
%
% Version 0.1
%
%%%

\section{Windows Applocker Disabled}
\label{15070}
%\addcontentsline{toc}{section}{\protect\numberline{}(Tool)}%

\openup 1em

\textbf{Full name:} Applocker disabled\hrulefill \\
%\textbf{Vendor/Provider:} (Vendor name)\hrulefill \\
%\textbf{URL: } \url{http://www.google.de} \hrulefill \\
{\bf Categorization:} \\

\openup -1em
\vspace{-3em}

% Please, write either Multi-OS, Linux, Windows, Mac
\begin{tabular}{p{.5\textwidth}p{.5\textwidth}}

\begin{todolist}
  	\item Multi-OS
	\item Linux
\end{todolist}
&
\begin{todolist}
	\item[\done] Windows
	\item macOS
\end{todolist}

\end{tabular}

\openup 1em

{\bf SMO products in scope:} \\

\openup -1em
\vspace{-3em}

\begin{tabular}{p{.5\textwidth}p{.5\textwidth}}

\begin{todolist}
  \item[\done] Generic
  \item[\done] VICOS OC 100
  \item[\done] VICOS S\&D
\end{todolist}
&
\begin{todolist}
  \item Rail9000
  \item DCS
  \item Others: \hrulefill
\end{todolist}

\end{tabular}

\openup 1em

{\bf Description:} \\

\openup -1em
\vspace{-2em}

%\begin{itemize}
%	\item (What does the rule do?)
%	\item (Why was the situation targeted (e.g., in a workshop)?)
%\end{itemize}

This rule identifies situations in which the Windows Windows Applocker (utility introduced with Microsoft Windows 7 operating systems) is disabled. AppLocker is an application that allows an administrator to restrict which programs users can execute based on several features (e.g., programs` paths, publishers, hashes, etc.) or with a custom Group Policy. Attackers might disable AppLocker to ease the execution of malicious program and, in general, impairing Windows cybersecurity defenses.

\openup 1em

{\bf Rule query:} \\

\openup -1em
\vspace{-2em}

\begin{lstlisting}[language=json,firstnumber=1]
{
  "query": {
    "bool": {
      "must": [
        {
          "match": {
            "eventId": "8008"
          }
        },
        {
          "match": {
            "cat": "winEvent"
          }
        },
        {
          "range": {
            "@timestamp": {
              "gt": "<bt>",
              "lt": "<et>"
            }
          }
        }
      ]
    }
  },
  "size": 1000
}
\end{lstlisting}

\openup 1em

{\bf Rule requirements:} \\

\openup -1em
\vspace{-2em}

Windows Applocker needs to be installed and running.

\openup 1em

{\bf Rule parameters:} \\

\openup -1em
\vspace{-2em}

\begin{itemize}
	\item A standard time frame defined by the ``greater than'' (gt) and ``less than'' parameters.
\end{itemize}

\openup 1em

{\bf Further information/categorization:} \\

\openup -1em
\vspace{-2em}

%\begin{itemize}
%	\item <Does the rule have an ATT\&CK label?>
%	\item <Might the rule create many false positives?>
%\end{itemize}

The rule can be labeled with the following MITRE ATT\&CK's tactics and techniques:
\begin{itemize}
	\item T1562 - ``Impair Defenses''
		\subitem{.001 - `` Disable or Modify Tools''}
\end{itemize}

\pagebreak


% !TEX root = ../AdditionalOsaRules.tex

%%%
%
% Version 0.1
%
%%%

\section{Windows Applocker Security Event}
\label{15071}
%\addcontentsline{toc}{section}{\protect\numberline{}(Tool)}%

\openup 1em

\textbf{Full name:} Applocker security event\hrulefill \\
%\textbf{Vendor/Provider:} (Vendor name)\hrulefill \\
%\textbf{URL: } \url{http://www.google.de} \hrulefill \\
{\bf Categorization:} \\

\openup -1em
\vspace{-3em}

% Please, write either Multi-OS, Linux, Windows, Mac
\begin{tabular}{p{.5\textwidth}p{.5\textwidth}}

\begin{todolist}
  	\item Multi-OS
	\item Linux
\end{todolist}
&
\begin{todolist}
	\item[\done] Windows
	\item macOS
\end{todolist}

\end{tabular}

\openup 1em

{\bf SMO products in scope:} \\

\openup -1em
\vspace{-3em}

\begin{tabular}{p{.5\textwidth}p{.5\textwidth}}

\begin{todolist}
  \item[\done] Generic
  \item[\done] VICOS OC 100
  \item[\done] VICOS S\&D
\end{todolist}
&
\begin{todolist}
  \item Rail9000
  \item DCS
  \item Others: \hrulefill
\end{todolist}

\end{tabular}

\openup 1em

{\bf Description:} \\

\openup -1em
\vspace{-2em}

%\begin{itemize}
%	\item (What does the rule do?)
%	\item (Why was the situation targeted (e.g., in a workshop)?)
%\end{itemize}

This rule report on security-relevant event triggered by the Windows AppLocker utility (introduced in Microsoft Windows 7 operating systems). AppLocker is an application that allows an administrator to restrict which programs users can execute based on several features (e.g., programs` paths, publishers, hashes, etc.) or with a custom Group Policy. The rule alerts on any error event notified by Applocker (see https://docs.microsoft.com/en-us/windows/security/threat-protection/windows-defender-application-control/applocker/using-event-viewer-with-applocker for more information).

\openup 1em

{\bf Rule query:} \\

\openup -1em
\vspace{-2em}

\begin{lstlisting}[language=json,firstnumber=1]
{
  "query": {
    "bool": {
      "must": [
        {
          "terms": {
            "eventId": [
              "8000",
              "8004",
              "8007",
              "8008",
              "8029",
              "8032",
              "8035",
              "8036",
              "8040"
            ]
          }
        },
        {
          "match": {
            "cat": "winEvent"
          }
        },
        {
          "range": {
            "@timestamp": {
              "gt": "<bt>",
              "lt": "<et>"
            }
          }
        }
      ]
    }
  },
  "size": 1000
}
\end{lstlisting}

\openup 1em

{\bf Rule requirements:} \\

\openup -1em
\vspace{-2em}

Windows Applocker needs to be installed and running.

\openup 1em

{\bf Rule parameters:} \\

\openup -1em
\vspace{-2em}

\begin{itemize}
	\item A standard time frame defined by the ``greater than'' (gt) and ``less than'' parameters.
\end{itemize}

\openup 1em

{\bf Further information/categorization:} \\

\openup -1em
\vspace{-2em}

%\begin{itemize}
%	\item <Does the rule have an ATT\&CK label?>
%	\item <Might the rule create many false positives?>
%\end{itemize}

None.

\pagebreak



% !TEX root = ../AdditionalOsaRules.tex

%%%
%
% Version 0.1
%
%%%

\section{Disk Space Exhaustion}
\label{15080}
%\addcontentsline{toc}{section}{\protect\numberline{}(Tool)}%

\openup 1em

\textbf{Full name:} Disk space exhaustion\hrulefill \\
%\textbf{Vendor/Provider:} (Vendor name)\hrulefill \\
%\textbf{URL: } \url{http://www.google.de} \hrulefill \\
{\bf Categorization:} \\

\openup -1em
\vspace{-3em}

% Please, write either Multi-OS, Linux, Windows, Mac
\begin{tabular}{p{.5\textwidth}p{.5\textwidth}}

\begin{todolist}
  	\item Multi-OS
	\item Linux
\end{todolist}
&
\begin{todolist}
	\item[\done] Windows
	\item macOS
\end{todolist}

\end{tabular}

\openup 1em

{\bf SMO products in scope:} \\

\openup -1em
\vspace{-3em}

\begin{tabular}{p{.5\textwidth}p{.5\textwidth}}

\begin{todolist}
  \item[\done] Generic
  \item[\done] VICOS OC 100
  \item[\done] VICOS S\&D
\end{todolist}
&
\begin{todolist}
  \item Rail9000
  \item DCS
  \item Others: \hrulefill
\end{todolist}

\end{tabular}

\openup 1em

{\bf Description:} \\

\openup -1em
\vspace{-2em}

%\begin{itemize}
%	\item (What does the rule do?)
%	\item (Why was the situation targeted (e.g., in a workshop)?)
%\end{itemize}

This rules identifies situations in which a Windows system has exhausted its space on disk. Such exhaustion can cause several problems to the system among which the impossibility to store security logs. This condition might be exploited by attackers to further hide their activities on the system.

\openup 1em

{\bf Rule query:} \\

\openup -1em
\vspace{-2em}

\begin{lstlisting}[language=json,firstnumber=1]
{
  "query": {
    "bool": {
      "must": [
        {
          "match": {
            "eventId": "2013"
          }
        },
        {
          "match": {
            "cat": "winEvent"
          }
        },
        {
          "range": {
            "@timestamp": {
              "gt": "<bt>",
              "lt": "<et>"
            }
          }
        }
      ]
    }
  },
  "size": 1000
}
\end{lstlisting}

\openup 1em

{\bf Rule requirements:} \\

\openup -1em
\vspace{-2em}

None.

\openup 1em

{\bf Rule parameters:} \\

\openup -1em
\vspace{-2em}

\begin{itemize}
	\item A standard time frame defined by the ``greater than'' (gt) and ``less than'' parameters.
\end{itemize}

\pagebreak



%%%%%%%%%
%%% SMO %%%
%%%%%%%%%

%%% Rail9000
% !TEX root = ../AdditionalOsaRules.tex

%%%
%
% Version 0.1
%
%%%

\section{Repeated Failed Login}
\label{15101}
%\addcontentsline{toc}{section}{\protect\numberline{}(Tool)}%

\openup 1em

\textbf{Full name:} Repeated failed login\hrulefill \\
%\textbf{Vendor/Provider:} (Vendor name)\hrulefill \\
%\textbf{URL: } \url{http://www.google.de} \hrulefill \\
{\bf Categorization:} \\

\openup -1em
\vspace{-3em}

% Please, write either Multi-OS, Linux, Windows, Mac
\begin{tabular}{p{.5\textwidth}p{.5\textwidth}}

\begin{todolist}
  	\item Multi-OS
	\item[\done] Linux
\end{todolist}
&
\begin{todolist}
	\item Windows
	\item macOS
\end{todolist}

\end{tabular}

\openup 1em

{\bf SMO products in scope:} \\

\openup -1em
\vspace{-3em}

\begin{tabular}{p{.5\textwidth}p{.5\textwidth}}

\begin{todolist}
  \item[\done] Generic
  \item VICOS OC 100
  \item VICOS S\&D
\end{todolist}
&
\begin{todolist}
  \item[\done] Rail9000
  \item DCS
  \item Others: \hrulefill
\end{todolist}

\end{tabular}

\openup 1em

{\bf Description:} \\

\openup -1em
\vspace{-2em}

%\begin{itemize}
%	\item (What does the rule do?)
%	\item (Why was the situation targeted (e.g., in a workshop)?)
%\end{itemize}

This rule identifies a situation in which several login attempts (happening within a specific time frame) are performed unsuccessfully.
According to the feedback collected during the Rail9000 workshop: ``as no feature exists on the application to notify (or even block) an account attempting to login repeatedly, this use case should be implement.''

\openup 1em

{\bf Rule query:} \\

\openup -1em
\vspace{-2em}

\begin{lstlisting}[language=json,firstnumber=1]
{
  "query": {
    "bool": {
      "must": [
        {
          "match_phrase": {
            "msg": "incorrect password attempts"
          }
        },
        {
          "bool": {
            "must_not": [
              {
                "match_phrase": {
                  "msg": "no new event"
                }
              }
            ]
          }
        },
        {
          "range": {
            "@timestamp": {
              "gt": "<bt>",
              "lt": "<et>"
            }
          }
        }
      ]
    }
  },
  "aggs": {
    "shost": {
      "terms": {
        "field": "shost.keyword"
      }
    }
  },
  "size": 1000
}
\end{lstlisting}

\openup 1em

{\bf Rule requirements:} \\

\openup -1em
\vspace{-2em}

This rule requires a Linux OS to log ``system'' and ``internal'' data.

\openup 1em

{\bf Rule parameters:} \\

\openup -1em
\vspace{-2em}

\begin{itemize}
	\item A standard time frame defined by the ``greater than'' (gt) and ``less than'' parameters.
\end{itemize}

\openup 1em

{\bf Further information/categorization:} \\

\openup -1em
\vspace{-2em}

%\begin{itemize}
%	\item <Does the rule have an ATT\&CK label?>
%	\item <Might the rule create many false positives?>
%\end{itemize}

The rule can be labeled with the following MITRE ATT\&CK's tactics and techniques:
\begin{itemize}
	\item T1110 - ``Brute Force''
		\subitem{.001 - `` Password Guessing''}
\end{itemize}

\pagebreak



%%% VICOS OC 100
% !TEX root = ../AdditionalOsaRules.tex

%%%
%
% Version 0.1
%
%%%

\section{Vicos Repeated Failed Login}
\label{15201}
%\addcontentsline{toc}{section}{\protect\numberline{}(Tool)}%

\openup 1em

\textbf{Full name:} Vicos repeated failed login\hrulefill \\
%\textbf{Vendor/Provider:} (Vendor name)\hrulefill \\
%\textbf{URL: } \url{http://www.google.de} \hrulefill \\
{\bf Categorization:} \\

\openup -1em
\vspace{-3em}

% Please, write either Multi-OS, Linux, Windows, Mac
\begin{tabular}{p{.5\textwidth}p{.5\textwidth}}

\begin{todolist}
  	\item Multi-OS
	\item Linux
\end{todolist}
&
\begin{todolist}
	\item[\done] Windows
	\item macOS
\end{todolist}

\end{tabular}

\openup 1em

{\bf SMO products in scope:} \\

\openup -1em
\vspace{-3em}

\begin{tabular}{p{.5\textwidth}p{.5\textwidth}}

\begin{todolist}
  \item Generic
  \item[\done] VICOS OC 100
  \item VICOS S\&D
\end{todolist}
&
\begin{todolist}
  \item Rail9000
  \item DCS
  \item Others: \hrulefill
\end{todolist}

\end{tabular}

\openup 1em

{\bf Description:} \\

\openup -1em
\vspace{-2em}

%\begin{itemize}
%	\item (What does the rule do?)
%	\item (Why was the situation targeted (e.g., in a workshop)?)
%\end{itemize}

This rule identifies situations in which a user unsuccessfully tries to login to either the VICOS OC 100 workstation, server or TTP module. Situations such as this one might indicate a malicious attempt to guess application credentials and gain unauthorized access to VICOS systems with a valid account. It is worth noting that, VICOS software provides a built-in feature to expand the time required for a login after a given number of failed attempts (thus, avoiding the possibility to actually gain vaild credentials by ``brute-forcing'' the interface). Nonetheless, the enablement of this feature is optional and left to the administrator. Whenever in use, the built-in feature, would make this OSA correlation rule unnecessary.

\openup 1em

{\bf Rule query:} \\

\openup -1em
\vspace{-2em}

\begin{lstlisting}[language=json,firstnumber=1]
{
  "query": {
    "bool": {
      "must": [
        {
          "bool": {
            "should": [
              {
                "match": {
                  "eventId": "152"
                }
              },
              {
                "match": {
                  "eventId": "552"
                }
              },
              {
                "match": {
                  "eventId": "1552"
                }
              }
            ]
          }
        },
        {
          "match": {
            "cat": "winEvent"
          }
        },
        {
          "match": {
            "channel": "Vicos OC 100 channel"
          }
        },
        {
          "range": {
            "@timestamp": {
              "gt": "<bt>",
              "lt": "<et>"
            }
          }
        }
      ]
    }
  },
  "aggs": {
    "shost": {
      "terms": {
        "field": "shost.keyword"
      }
    }
  },
  "size": 1000
}
\end{lstlisting}

\openup 1em

{\bf Rule requirements:} \\

\openup -1em
\vspace{-2em}

This rule requires VICOS software to log security events in the custom Windows event channel (the newest Vicos logging concept)

\openup 1em

{\bf Rule parameters:} \\

\openup -1em
\vspace{-2em}

\begin{itemize}
	\item A standard time frame defined by the ``greater than'' (gt) and ``less than'' parameters.
\end{itemize}

\openup 1em

{\bf Further information/categorization:} \\

\openup -1em
\vspace{-2em}

%\begin{itemize}
%	\item <Does the rule have an ATT\&CK label?>
%	\item <Might the rule create many false positives?>
%\end{itemize}

The rule can be labeled with the following MITRE ATT\&CK's tactics and techniques:
\begin{itemize}
	\item T1110 - ``Brute Force''
		\subitem{.001 - `` Password Guessing''}
\end{itemize}

\pagebreak


% !TEX root = ../AdditionalOsaRules.tex

%%%
%
% Version 0.1
%
%%%

\section{Vicos Account Deletion}
\label{15202}
%\addcontentsline{toc}{section}{\protect\numberline{}(Tool)}%

\openup 1em

\textbf{Full name:} Vicos account deletion\hrulefill \\
%\textbf{Vendor/Provider:} (Vendor name)\hrulefill \\
%\textbf{URL: } \url{http://www.google.de} \hrulefill \\
{\bf Categorization:} \\

\openup -1em
\vspace{-3em}

% Please, write either Multi-OS, Linux, Windows, Mac
\begin{tabular}{p{.5\textwidth}p{.5\textwidth}}

\begin{todolist}
  	\item Multi-OS
	\item Linux
\end{todolist}
&
\begin{todolist}
	\item[\done] Windows
	\item macOS
\end{todolist}

\end{tabular}

\openup 1em

{\bf SMO products in scope:} \\

\openup -1em
\vspace{-3em}

\begin{tabular}{p{.5\textwidth}p{.5\textwidth}}

\begin{todolist}
  \item Generic
  \item[\done] VICOS OC 100
  \item VICOS S\&D
\end{todolist}
&
\begin{todolist}
  \item Rail9000
  \item DCS
  \item Others: \hrulefill
\end{todolist}

\end{tabular}

\openup 1em

{\bf Description:} \\

\openup -1em
\vspace{-2em}

%\begin{itemize}
%	\item (What does the rule do?)
%	\item (Why was the situation targeted (e.g., in a workshop)?)
%\end{itemize}

This rule identifies situations in which a valid user account is deleted from either the VICOS OC 100 workstation, server or TTP module. Situations such as this one might indicate a malicious attempt to prevent authorized operators to access the VICOS interfaces and thus perform their normal operations (e.g., track monitoring).

\openup 1em

{\bf Rule query:} \\

\openup -1em
\vspace{-2em}

\begin{lstlisting}[language=json,firstnumber=1]
{
  "query": {
    "bool": {
      "must": [
        {
          "bool": {
            "should": [
              {
                "match": {
                  "eventId": "155"
                }
              },
              {
                "match": {
                  "eventId": "555"
                }
              },
              {
                "match": {
                  "eventId": "1555"
                }
              }
            ]
          }
        },
        {
          "match": {
            "cat": "winEvent"
          }
        },
        {
          "match": {
            "channel": "Vicos OC 100 channel"
          }
        },
        {
          "range": {
            "@timestamp": {
              "gt": "<bt>",
              "lt": "<et>"
            }
          }
        }
      ]
    }
  },
  "size": 1000
}
\end{lstlisting}

\openup 1em

{\bf Rule requirements:} \\

\openup -1em
\vspace{-2em}

This rule requires VICOS software to log security events in the custom Windows event channel (the newest Vicos logging concept)

\openup 1em

{\bf Rule parameters:} \\

\openup -1em
\vspace{-2em}

\begin{itemize}
	\item A standard time frame defined by the ``greater than'' (gt) and ``less than'' parameters.
\end{itemize}

\openup 1em

{\bf Further information/categorization:} \\

\openup -1em
\vspace{-2em}

%\begin{itemize}
%	\item <Does the rule have an ATT\&CK label?>
%	\item <Might the rule create many false positives?>
%\end{itemize}

The rule can be labeled with the following MITRE ATT\&CK's tactics and techniques:
\begin{itemize}
	\item T1531 - ``Account Access Removal''
\end{itemize}

\pagebreak


% !TEX root = ../AdditionalOsaRules.tex

%%%
%
% Version 0.1
%
%%%

\section{Vicos Login Attempt Of Non Authorized Username}
\label{15203}
%\addcontentsline{toc}{section}{\protect\numberline{}(Tool)}%

\openup 1em

\textbf{Full name:} Vicos login attempt of non authorized username\hrulefill \\
%\textbf{Vendor/Provider:} (Vendor name)\hrulefill \\
%\textbf{URL: } \url{http://www.google.de} \hrulefill \\
{\bf Categorization:} \\

\openup -1em
\vspace{-3em}

% Please, write either Multi-OS, Linux, Windows, Mac
\begin{tabular}{p{.5\textwidth}p{.5\textwidth}}

\begin{todolist}
  	\item Multi-OS
	\item Linux
\end{todolist}
&
\begin{todolist}
	\item[\done] Windows
	\item macOS
\end{todolist}

\end{tabular}

\openup 1em

{\bf SMO products in scope:} \\

\openup -1em
\vspace{-3em}

\begin{tabular}{p{.5\textwidth}p{.5\textwidth}}

\begin{todolist}
  \item Generic
  \item[\done] VICOS OC 100
  \item VICOS S\&D
\end{todolist}
&
\begin{todolist}
  \item Rail9000
  \item DCS
  \item Others: \hrulefill
\end{todolist}

\end{tabular}

\openup 1em

{\bf Description:} \\

\openup -1em
\vspace{-2em}

%\begin{itemize}
%	\item (What does the rule do?)
%	\item (Why was the situation targeted (e.g., in a workshop)?)
%\end{itemize}

This rule identifies situations in which a non-allowlisted user account successfully logins to either the VICOS OC 100 workstation, server or TTP module. Situations such as this one might indicate a discrepancy between the known accounts (the one allowlisted) and the actual accounts capable of logging in. In some cases, new accounts can be artificiously created (e.g., not through the available interface but directly added to the list of available users) to maintain access to the a given application.

\openup 1em

{\bf Rule query:} \\

\openup -1em
\vspace{-2em}

\begin{lstlisting}[language=json,firstnumber=1]
{
  "query": {
    "bool": {
      "must": [
        {
          "bool": {
            "should": [
              {
                "match": {
                  "eventId": "150"
                }
              },
              {
                "match": {
                  "eventId": "550"
                }
              },
              {
                "match": {
                  "eventId": "1550"
                }
              }
            ]
          }
        },
        {
          "bool": {
            "must_not": [
              {
                "terms": {
                  "message": [
                    "username1",
                    "username2",
                    "..."
                  ]
                }
              }
            ]
          }
        },
        {
          "match": {
            "cat": "winEvent"
          }
        },
        {
          "match": {
            "channel": "Vicos OC 100 channel"
          }
        },
        {
          "range": {
            "@timestamp": {
              "gt": "<bt>",
              "lt": "<et>"
            }
          }
        }
      ]
    }
  },
  "size": 1000
}
\end{lstlisting}

\openup 1em

{\bf Rule requirements:} \\

\openup -1em
\vspace{-2em}

This rule requires VICOS software to log security events in the custom Windows event channel (the newest Vicos logging concept). Furthermore, this rule assumes the presence of a well-defined list of allowlisted usernames.

\openup 1em

{\bf Rule parameters:} \\

\openup -1em
\vspace{-2em}

\begin{itemize}
	\item A standard time frame defined by the ``greater than'' (gt) and ``less than'' parameters.
	\item \emph{usernameX}: A allowlisted username
\end{itemize}

\openup 1em

{\bf Further information/categorization:} \\

\openup -1em
\vspace{-2em}

%\begin{itemize}
%	\item <Does the rule have an ATT\&CK label?>
%	\item <Might the rule create many false positives?>
%\end{itemize}

The rule can be labeled with the following MITRE ATT\&CK's tactics and techniques:
\begin{itemize}
	\item T1078 - ``Valid Accounts''
		\subitem{.003 - `` Local Accounts''}
\end{itemize}

\pagebreak


% !TEX root = ../AdditionalOsaRules.tex

%%%
%
% Version 0.1
%
%%%

\section{Vicos Login Logout At Unusual Time}
\label{15204}
%\addcontentsline{toc}{section}{\protect\numberline{}(Tool)}%

\openup 1em

\textbf{Full name:} Vicos login logout at unusual time\hrulefill \\
%\textbf{Vendor/Provider:} (Vendor name)\hrulefill \\
%\textbf{URL: } \url{http://www.google.de} \hrulefill \\
{\bf Categorization:} \\

\openup -1em
\vspace{-3em}

% Please, write either Multi-OS, Linux, Windows, Mac
\begin{tabular}{p{.5\textwidth}p{.5\textwidth}}

\begin{todolist}
  	\item Multi-OS
	\item Linux
\end{todolist}
&
\begin{todolist}
	\item[\done] Windows
	\item macOS
\end{todolist}

\end{tabular}

\openup 1em

{\bf SMO products in scope:} \\

\openup -1em
\vspace{-3em}

\begin{tabular}{p{.5\textwidth}p{.5\textwidth}}

\begin{todolist}
  \item Generic
  \item[\done] VICOS OC 100
  \item VICOS S\&D
\end{todolist}
&
\begin{todolist}
  \item Rail9000
  \item DCS
  \item Others: \hrulefill
\end{todolist}

\end{tabular}

\openup 1em

{\bf Description:} \\

\openup -1em
\vspace{-2em}

%\begin{itemize}
%	\item (What does the rule do?)
%	\item (Why was the situation targeted (e.g., in a workshop)?)
%\end{itemize}

This rule identifies situations in which login and logout operations to either the VICOS OC 100 workstation, server or TTP module, are performed or attempted outside the normal working hours (e.g., whenever operators work in shifts, this rule can ensure that such login and logout operations happen in the supposed timeframe). As attackers can try to take advantage of the absence of personnel to exploit the target systems, rules like these one work on restricting the categories of security-related events happening at unexpected time.

\openup 1em

{\bf Rule query:} \\

\openup -1em
\vspace{-2em}

\begin{lstlisting}[language=json,firstnumber=1]
{
  "runtime_mappings": {
    "daytime": {
      "type": "long",
      "script": {
        "source": "emit(doc['@timestamp'].value.toEpochSecond()%86400)"
      }
    }
  },
  "query": {
    "bool": {
      "must": [
        {
          "bool": {
            "should": [
              {
                "match": {
                  "eventId": "150"
                }
              },
              {
                "match": {
                  "eventId": "550"
                }
              },
              {
                "match": {
                  "eventId": "1550"
                }
              },
              {
                "match": {
                  "eventId": "152"
                }
              },
              {
                "match": {
                  "eventId": "552"
                }
              },
              {
                "match": {
                  "eventId": "1552"
                }
              }
            ]
          }
        },
        {
          "match": {
            "cat": "winEvent"
          }
        },
        {
          "bool": {
            "should": [
              {
                "range": {
                  "daytime": {
                    "lte": "shitfStartingTime"
                  }
                }
              },
              {
                "range": {
                  "daytime": {
                    "gte": "shitfFinishingTime"
                  }
                }
              },
              {
                "range": {
                  "@timestamp": {
                    "gt": "<bt>",
                    "lt": "<et>"
                  }
                }
              }
            ]
          }
        }
      ]
    }
  },
  "fields": [
    "daytime"
  ]
}
\end{lstlisting}

\openup 1em

{\bf Rule requirements:} \\

\openup -1em
\vspace{-2em}

This rule requires VICOS software to log security events in the custom Windows event channel (the newest Vicos logging concept). Furthermore, this rule assumes the definition of a working shift.

\openup 1em

{\bf Rule parameters:} \\

\openup -1em
\vspace{-2em}

\begin{itemize}
	\item A standard time frame defined by the ``greater than'' (gt) and ``less than'' parameters.
	\item \emph{shitfStartingTime}: The number of seconds from midnights indicating the normal shift starting time (e.g., 28800 would indicate ``8:00'')
	\item \emph{shitfFinishingTime}: The number of seconds from midnights indicating the normal shift ending time (e.g., 64800 would indicate ``18:00'')
\end{itemize}

\openup 1em

{\bf Further information/categorization:} \\

\openup -1em
\vspace{-2em}

%\begin{itemize}
%	\item <Does the rule have an ATT\&CK label?>
%	\item <Might the rule create many false positives?>
%\end{itemize}

The rule can be labeled with the following MITRE ATT\&CK's tactics and techniques:
\begin{itemize}
	\item T1078 - ``Valid Accounts''
		\subitem{.003 - `` Local Accounts''}
\end{itemize}

\pagebreak


% !TEX root = ../AdditionalOsaRules.tex

%%%
%
% Version 0.1
%
%%%

\section{Vicos Critical Account Creation}
\label{15205}
%\addcontentsline{toc}{section}{\protect\numberline{}(Tool)}%

\openup 1em

\textbf{Full name:} Vicos critical account creation\hrulefill \\
%\textbf{Vendor/Provider:} (Vendor name)\hrulefill \\
%\textbf{URL: } \url{http://www.google.de} \hrulefill \\
{\bf Categorization:} \\

\openup -1em
\vspace{-3em}

% Please, write either Multi-OS, Linux, Windows, Mac
\begin{tabular}{p{.5\textwidth}p{.5\textwidth}}

\begin{todolist}
  	\item Multi-OS
	\item Linux
\end{todolist}
&
\begin{todolist}
	\item[\done] Windows
	\item macOS
\end{todolist}

\end{tabular}

\openup 1em

{\bf SMO products in scope:} \\

\openup -1em
\vspace{-3em}

\begin{tabular}{p{.5\textwidth}p{.5\textwidth}}

\begin{todolist}
  \item Generic
  \item[\done] VICOS OC 100
  \item VICOS S\&D
\end{todolist}
&
\begin{todolist}
  \item Rail9000
  \item DCS
  \item Others: \hrulefill
\end{todolist}

\end{tabular}

\openup 1em

{\bf Description:} \\

\openup -1em
\vspace{-2em}

%\begin{itemize}
%	\item (What does the rule do?)
%	\item (Why was the situation targeted (e.g., in a workshop)?)
%\end{itemize}

This rule identifies situations in which a user account is created in a VICOS OC 100 server (such an account misbehaving might be more impactful than the ones created in a workstation or in a TTP)

\openup 1em

{\bf Rule query:} \\

\openup -1em
\vspace{-2em}

\begin{lstlisting}[language=json,firstnumber=1]
{
  "query": {
    "bool": {
      "must": [
        {
          "match": {
            "eventId": "1553"
          }
        },
        {
          "match": {
            "cat": "winEvent"
          }
        },
        {
          "match": {
            "channel": "Vicos OC 100 channel"
          }
        },
        {
          "range": {
            "@timestamp": {
              "gt": "<bt>",
              "lt": "<et>"
            }
          }
        }
      ]
    }
  },
  "size": 1000
}
\end{lstlisting}

\openup 1em

{\bf Rule requirements:} \\

\openup -1em
\vspace{-2em}

This rule requires VICOS software to log security events in the custom Windows event channel (the newest Vicos logging concept)

\openup 1em

{\bf Rule parameters:} \\

\openup -1em
\vspace{-2em}

\begin{itemize}
	\item A standard time frame defined by the ``greater than'' (gt) and ``less than'' parameters.
\end{itemize}

\openup 1em

{\bf Further information/categorization:} \\

\openup -1em
\vspace{-2em}

%\begin{itemize}
%	\item <Does the rule have an ATT\&CK label?>
%	\item <Might the rule create many false positives?>
%\end{itemize}

The rule can be labeled with the following MITRE ATT\&CK's tactics and techniques:
\begin{itemize}
	\item T1136 - ``Create Account''
		\subitem{.001 - `` Local Account''}
\end{itemize}

\pagebreak


% !TEX root = ../AdditionalOsaRules.tex

%%%
%
% Version 0.1
%
%%%

\section{Vicos Operation Performed At Anomalous Time}
\label{15206}
%\addcontentsline{toc}{section}{\protect\numberline{}(Tool)}%

\openup 1em

\textbf{Full name:} Vicos operation performed at anomalous time\hrulefill \\
%\textbf{Vendor/Provider:} (Vendor name)\hrulefill \\
%\textbf{URL: } \url{http://www.google.de} \hrulefill \\
{\bf Categorization:} \\

\openup -1em
\vspace{-3em}

% Please, write either Multi-OS, Linux, Windows, Mac
\begin{tabular}{p{.5\textwidth}p{.5\textwidth}}

\begin{todolist}
  	\item Multi-OS
	\item Linux
\end{todolist}
&
\begin{todolist}
	\item[\done] Windows
	\item macOS
\end{todolist}

\end{tabular}

\openup 1em

{\bf SMO products in scope:} \\

\openup -1em
\vspace{-3em}

\begin{tabular}{p{.5\textwidth}p{.5\textwidth}}

\begin{todolist}
  \item Generic
  \item[\done] VICOS OC 100
  \item VICOS S\&D
\end{todolist}
&
\begin{todolist}
  \item Rail9000
  \item DCS
  \item Others: \hrulefill
\end{todolist}

\end{tabular}

\openup 1em

{\bf Description:} \\

\openup -1em
\vspace{-2em}

%\begin{itemize}
%	\item (What does the rule do?)
%	\item (Why was the situation targeted (e.g., in a workshop)?)
%\end{itemize}

This rule identifies situations in which a an operation is performed in a VICOS system outside working hours. As for rule 15022 (``Login Logout At Unusual Time'), this rule might be helpful in cases where normal working hours are well defined or operators work in shifts. As attackers can try to take advantage of the absence of personnel to exploit the target systems, rules like these one work on restricting the categories of security-related events happening at unexpected time.

\openup 1em

{\bf Rule query:} \\

\openup -1em
\vspace{-2em}

\begin{lstlisting}[language=json,firstnumber=1]
{
  "runtime_mappings": {
    "daytime": {
      "type": "long",
      "script": {
        "source": "emit(doc['@timestamp'].value.toEpochSecond()%86400)"
      }
    }
  },
  "query": {
    "bool": {
      "must": [
        {
          "terms": {
            "eventId": [
              "Event1",
              "Event2",
              "..."
            ]
          }
        },
        {
          "match": {
            "cat": "winEvent"
          }
        },
        {
          "match": {
            "channel": "Vicos OC 100 channel"
          }
        },
        {
          "bool": {
            "should": [
              {
                "range": {
                  "daytime": {
                    "lte": 28800
                  }
                }
              },
              {
                "range": {
                  "daytime": {
                    "gte": 64800
                  }
                }
              }
            ]
          }
        },
        {
          "range": {
            "@timestamp": {
              "gt": "<bt>",
              "lt": "<et>"
            }
          }
        }
      ]
    }
  },
  "fields": [
    "daytime"
  ]
}
\end{lstlisting}

\openup 1em

{\bf Rule requirements:} \\

\openup -1em
\vspace{-2em}

This rule requires VICOS software to log security events in the custom Windows event channel (the newest Vicos logging concept)

\openup 1em

{\bf Rule parameters:} \\

\openup -1em
\vspace{-2em}

\begin{itemize}
	\item A standard time frame defined by the ``greater than'' (gt) and ``less than'' parameters.
	\item \emph{EventX}: The ID of the operation of interest (taken from the ``Vicos Security Log Manifest'')
\end{itemize}

\openup 1em

{\bf Further information/categorization:} \\

\openup -1em
\vspace{-2em}

%\begin{itemize}
%	\item <Does the rule have an ATT\&CK label?>
%	\item <Might the rule create many false positives?>
%\end{itemize}

None.

\pagebreak


% !TEX root = ../AdditionalOsaRules.tex

%%%
%
% Version 0.1
%
%%%

\section{Vicos Application Log Cleared}
\label{15207}
%\addcontentsline{toc}{section}{\protect\numberline{}(Tool)}%

\openup 1em

\textbf{Full name:} Vicos application log cleared\hrulefill \\
%\textbf{Vendor/Provider:} (Vendor name)\hrulefill \\
%\textbf{URL: } \url{http://www.google.de} \hrulefill \\
{\bf Categorization:} \\

\openup -1em
\vspace{-3em}

% Please, write either Multi-OS, Linux, Windows, Mac
\begin{tabular}{p{.5\textwidth}p{.5\textwidth}}

\begin{todolist}
  	\item Multi-OS
	\item Linux
\end{todolist}
&
\begin{todolist}
	\item[\done] Windows
	\item macOS
\end{todolist}

\end{tabular}

\openup 1em

{\bf SMO products in scope:} \\

\openup -1em
\vspace{-3em}

\begin{tabular}{p{.5\textwidth}p{.5\textwidth}}

\begin{todolist}
  \item Generic
  \item[\done] VICOS OC 100
  \item VICOS S\&D
\end{todolist}
&
\begin{todolist}
  \item Rail9000
  \item DCS
  \item Others: \hrulefill
\end{todolist}

\end{tabular}

\openup 1em

{\bf Description:} \\

\openup -1em
\vspace{-2em}

%\begin{itemize}
%	\item (What does the rule do?)
%	\item (Why was the situation targeted (e.g., in a workshop)?)
%\end{itemize}

This rule raise an alert whenever the VICOS OC 100 log is cleared. Attackers might try to delete logs to cover their actions and make forensics and recovery activities harder for the security operators. For old versions of VICOS the same misbehavior might be monitored with rule 15028 (``Windows Object Deletion')

\openup 1em

{\bf Rule query:} \\

\openup -1em
\vspace{-2em}

\begin{lstlisting}[language=json,firstnumber=1]
{
  "query": {
    "bool": {
      "must": [
        {
          "bool": {
            "should": [
              {
                "match": {
                  "eventId": "1102"
                }
              },
              {
                "match": {
                  "eventId": "517"
                }
              }
            ]
          }
        },
        {
          "match": {
            "channel": "Vicos OC 100 channel"
          }
        },
        {
          "match": {
            "cat": "winEvent"
          }
        },
        {
          "range": {
            "@timestamp": {
              "gt": "<bt>",
              "lt": "<et>"
            }
          }
        }
      ]
    }
  },
  "size": 1000
}
\end{lstlisting}

\openup 1em

{\bf Rule requirements:} \\

\openup -1em
\vspace{-2em}

This rule requires VICOS software to log security events in the custom Windows event channel (the newest Vicos logging concept)

\openup 1em

{\bf Rule parameters:} \\

\openup -1em
\vspace{-2em}

\begin{itemize}
	\item A standard time frame defined by the ``greater than'' (gt) and ``less than'' parameters.
\end{itemize}

\openup 1em

{\bf Further information/categorization:} \\

\openup -1em
\vspace{-2em}

%\begin{itemize}
%	\item <Does the rule have an ATT\&CK label?>
%	\item <Might the rule create many false positives?>
%\end{itemize}

The rule can be labeled with the following MITRE ATT\&CK's tactics and techniques:
\begin{itemize}
	\item T1070 - ``Indicator Removal on Host''
		\subitem{.001 - `` Clear Windows Event Logs''}
\end{itemize}

\pagebreak



%%% DCS
% !TEX root = ../AdditionalOsaRules.tex

%%%
%
% Version 0.1
%
%%%

\section{Dcu Invalid Signature}
\label{15301}
%\addcontentsline{toc}{section}{\protect\numberline{}(Tool)}%

\openup 1em

\textbf{Full name:} Dcu invalid signature\hrulefill \\
%\textbf{Vendor/Provider:} (Vendor name)\hrulefill \\
%\textbf{URL: } \url{http://www.google.de} \hrulefill \\
{\bf Categorization:} \\

\openup -1em
\vspace{-3em}

% Please, write either Multi-OS, Linux, Windows, Mac
\begin{tabular}{p{.5\textwidth}p{.5\textwidth}}

\begin{todolist}
  	\item Multi-OS
	\item[\done] Linux
\end{todolist}
&
\begin{todolist}
	\item Windows
	\item macOS
\end{todolist}

\end{tabular}

\openup 1em

{\bf SMO products in scope:} \\

\openup -1em
\vspace{-3em}

\begin{tabular}{p{.5\textwidth}p{.5\textwidth}}

\begin{todolist}
  \item Generic
  \item VICOS OC 100
  \item VICOS S\&D
\end{todolist}
&
\begin{todolist}
  \item Rail9000
  \item DCS
  \item Others: \bf{DCU} \hrulefill
\end{todolist}

\end{tabular}

\openup 1em

{\bf Description:} \\

\openup -1em
\vspace{-2em}

%\begin{itemize}
%	\item (What does the rule do?)
%	\item (Why was the situation targeted (e.g., in a workshop)?)
%\end{itemize}

This rule identifies situation in which a digital signature needed by the DCU is not valid. Such situations might arise when the DCU tries to validate digital certificates included in specific files (e.g., XML configuration files)

\openup 1em

{\bf Rule query:} \\

\openup -1em
\vspace{-2em}

\begin{lstlisting}[language=json,firstnumber=1]
{
  "query": {
    "bool": {
      "must": [
        {
          "match_phrase": {
            "msg": "invalid signature"
          }
        },
        {
          "bool": {
            "must_not": [
              {
                "match_phrase": {
                  "msg": "no new event"
                }
              }
            ]
          }
        },
        {
          "range": {
            "@timestamp": {
              "gt": "<bt>",
              "lt": "<et>"
            }
          }
        }
      ]
    }
  },
  "size": 1000
}
\end{lstlisting}

\openup 1em

{\bf Rule requirements:} \\

\openup -1em
\vspace{-2em}

None.

\openup 1em

{\bf Rule parameters:} \\

\openup -1em
\vspace{-2em}

\begin{itemize}
	\item A standard time frame defined by the ``greater than'' (gt) and ``less than'' parameters.
\end{itemize}

\openup 1em

{\bf Further information/categorization:} \\

\openup -1em
\vspace{-2em}

None.

\pagebreak
\input{rules/15401_EsxiAuthenticationFailure}
% !TEX root = ../AdditionalOsaRules.tex

%%%
%
% Version 0.1
%
%%%

\section{Esxi SSH Enabled}
%\addcontentsline{toc}{section}{\protect\numberline{}(Tool)}%

\openup 1em

\textbf{Full name:} Esxi SSH enabled\hrulefill \\
%\textbf{Vendor/Provider:} (Vendor name)\hrulefill \\
%\textbf{URL: } \url{http://www.google.de} \hrulefill \\
{\bf Categorization:} \\

\openup -1em
\vspace{-3em}

% Please, write either Multi-OS, Linux, Windows, Mac
\begin{tabular}{p{.5\textwidth}p{.5\textwidth}}

\begin{todolist}
  	\item Multi-OS
	\item[\done] Linux
\end{todolist}
&
\begin{todolist}
	\item Windows
	\item macOS
\end{todolist}

\end{tabular}

\openup 1em

{\bf SMO products in scope:} \\

\openup -1em
\vspace{-3em}

\begin{tabular}{p{.5\textwidth}p{.5\textwidth}}

\begin{todolist}
  \item Generic
  \item VICOS OC 100
  \item VICOS S\&D
\end{todolist}
&
\begin{todolist}
  \item Rail9000
  \item DCS
  \item Others: \bf{ESXi} \hrulefill
\end{todolist}

\end{tabular}

\openup 1em

{\bf Description:} \\

\openup -1em
\vspace{-2em}

%\begin{itemize}
%	\item (What does the rule do?)
%	\item (Why was the situation targeted (e.g., in a workshop)?)
%\end{itemize}

This rule identifies situation in which SSH access to the ESXi hypervisor is enabled. In certain situations, SSH access to the ESXi hypervisor might be disabled to avoid remote connections and protect the resource. Any unwanted change to this configuration should be alerted as sign of possible intrusions and attempts to weaken hypervisor`s perimeter security.

\openup 1em

{\bf Rule query:} \\

\openup -1em
\vspace{-2em}

\begin{lstlisting}[language=json,firstnumber=1]
{
  "query": {
    "bool": {
      "must": [
        {
          "match_phrase": {
            "msg": "SSH login enabled"
          }
        },
        {
          "bool": {
            "must_not": [
              {
                "match_phrase": {
                  "msg": "no new event"
                }
              }
            ]
          }
        },
        {
          "range": {
            "@timestamp": {
              "gt": "<bt>",
              "lt": "<et>"
            }
          }
        }
      ]
    }
  },
  "size": 1000
}
\end{lstlisting}

\openup 1em

{\bf Rule requirements:} \\

\openup -1em
\vspace{-2em}

None.

\openup 1em

{\bf Rule parameters:} \\

\openup -1em
\vspace{-2em}

\begin{itemize}
	\item A standard time frame defined by the ``greater than'' (gt) and ``less than'' parameters.
\end{itemize}

\openup 1em

{\bf Further information/categorization:} \\

\openup -1em
\vspace{-2em}

None.

\pagebreak


% !TEX root = ../AdditionalOsaRules.tex

%%%
%
% Version 0.1
%
%%%

\section{Login Hirschmann Switch (DRAFT)}
\label{15701}
%\addcontentsline{toc}{section}{\protect\numberline{}(Tool)}%

\openup 1em

\textbf{Full name:} Login hirschmann switch\hrulefill \\
%\textbf{Vendor/Provider:} (Vendor name)\hrulefill \\
%\textbf{URL: } \url{http://www.google.de} \hrulefill \\
{\bf Categorization:} \\

\openup -1em
\vspace{-3em}

% Please, write either Multi-OS, Linux, Windows, Mac
\begin{tabular}{p{.5\textwidth}p{.5\textwidth}}

\begin{todolist}
  	\item Multi-OS
	\item Linux
\end{todolist}
&
\begin{todolist}
	\item Windows
	\item macOS
\end{todolist}

\end{tabular}

\openup 1em

{\bf SMO products in scope:} \\

\openup -1em
\vspace{-3em}

\begin{tabular}{p{.5\textwidth}p{.5\textwidth}}

\begin{todolist}
  \item Generic
  \item VICOS OC 100
  \item VICOS S\&D
\end{todolist}
&
\begin{todolist}
  \item Rail9000
  \item[\done] DCS
  \item Others: Hirschmann Switch\hrulefill
\end{todolist}

\end{tabular}

\openup 1em

{\bf Description:} \\

\openup -1em
\vspace{-2em}

%\begin{itemize}
%	\item (What does the rule do?)
%	\item (Why was the situation targeted (e.g., in a workshop)?)
%\end{itemize}

This rule identifies a new login to the Hirschmann switches by matching phrase in Hirschmann switch log.

\openup 1em

{\bf Rule query:} \\

\openup -1em
\vspace{-2em}

\begin{lstlisting}[language=json,firstnumber=1]
{
  "query": {
    "bool": {
      "must": [
        {
          "match_phrase": {
            "msg": "S_web_HTTPSNMP_LOGIN_USER_SUCCESS"
          }
        },
        {
          "bool": {
            "must_not": [
              {
                "match_phrase": {
                  "msg": "no new event"
                }
              }
            ]
          }
        },
        {
          "range": {
            "@timestamp": {
              "gt": "<bt>",
              "lt": "<et>"
            }
          }
        }
      ]
    }
  },
  "size": 1000
}
\end{lstlisting}

\openup 1em

{\bf Rule requirements:} \\

\openup -1em
\vspace{-2em}

None.

\openup 1em

{\bf Rule parameters:} \\

\openup -1em
\vspace{-2em}

\begin{itemize}
	\item A standard time frame defined by the ``greater than'' (gt) and ``less than'' parameters.
\end{itemize}

\openup 1em

{\bf Further information/categorization:} \\

\openup -1em
\vspace{-2em}

%\begin{itemize}
%	\item <Does the rule have an ATT\&CK label?>
%	\item <Might the rule create many false positives?>
%\end{itemize}

None.

\pagebreak


% !TEX root = ../AdditionalOsaRules.tex

%%%
%
% Version 0.1
%
%%%

\section{Fortigate Firewall Admin Login (DRAFT)}
\label{15800}
%\addcontentsline{toc}{section}{\protect\numberline{}(Tool)}%

\openup 1em

\textbf{Full name:} Fortigate firewall admin login \hrulefill \\
%\textbf{Vendor/Provider:} (Vendor name)\hrulefill \\
%\textbf{URL: } \url{http://www.google.de} \hrulefill \\
{\bf Categorization:} \\

\openup -1em
\vspace{-3em}

% Please, write either Multi-OS, Linux, Windows, Mac
\begin{tabular}{p{.5\textwidth}p{.5\textwidth}}

\begin{todolist}
  	\item Multi-OS
	\item Linux
\end{todolist}
&
\begin{todolist}
	\item Windows
	\item macOS
\end{todolist}

\end{tabular}

\openup 1em

{\bf SMO products in scope:} \\

\openup -1em
\vspace{-3em}

\begin{tabular}{p{.5\textwidth}p{.5\textwidth}}

\begin{todolist}
  \item Generic
  \item VICOS OC 100
  \item VICOS S\&D
\end{todolist}
&
\begin{todolist}
  \item Rail9000
  \item[\done] DCS
  \item Others: Fortigate Firewall\hrulefill
\end{todolist}

\end{tabular}

\openup 1em

{\bf Description:} \\

\openup -1em
\vspace{-2em}

%\begin{itemize}
%	\item (What does the rule do?)
%	\item (Why was the situation targeted (e.g., in a workshop)?)
%\end{itemize}

This rule detects the successful or failed login of an administrator to the FortiGate firewall

\openup 1em

{\bf Rule query:} \\

\openup -1em
\vspace{-2em}

\begin{lstlisting}[language=json,firstnumber=1]
{
  "query": {
    "bool": {
      "must": [
        {
          "bool": {
            "should": [
              {
                "wildcard": {
                  "fortigateLogid": "????032001"
                }
              },
              {
                "wildcard": {
                  "fortigateLogid": "????032002"
                }
              }
            ]
          }
        },
        {
          "range": {
            "@timestamp": {
              "gte": "<bt>",
              "lt": "<et>"
            }
          }
        }
      ]
    }
  },
  "size": 1000
}
\end{lstlisting}

\openup 1em

{\bf Rule requirements:} \\

\openup -1em
\vspace{-2em}

The FortiGate firewall sends syslogs to OSA and Logstash is configured appropriately.

\openup 1em

{\bf Rule parameters:} \\

\openup -1em
\vspace{-2em}

\begin{itemize}
	\item A standard time frame defined by the ``greater than'' (gt) and ``less than'' parameters.
\end{itemize}

\openup 1em

{\bf Further information/categorization:} \\

\openup -1em
\vspace{-2em}

%\begin{itemize}
%	\item <Does the rule have an ATT\&CK label?>
%	\item <Might the rule create many false positives?>
%\end{itemize}

The rule can be labeled with the following MITRE ATT\&CK's tactics and techniques:
\begin{itemize}
	\item T1078 - ``Valid Accounts''
\end{itemize}

\pagebreak


% !TEX root = ../AdditionalOsaRules.tex

%%%
%
% Version 0.1
%
%%%

\section{Fortigate Firewall U T M Security Event (DRAFT)}
\label{15810}
%\addcontentsline{toc}{section}{\protect\numberline{}(Tool)}%

\openup 1em

\textbf{Full name:} Fortigate firewall u t m security event \hrulefill \\
%\textbf{Vendor/Provider:} (Vendor name)\hrulefill \\
%\textbf{URL: } \url{http://www.google.de} \hrulefill \\
{\bf Categorization:} \\

\openup -1em
\vspace{-3em}

% Please, write either Multi-OS, Linux, Windows, Mac
\begin{tabular}{p{.5\textwidth}p{.5\textwidth}}

\begin{todolist}
  	\item Multi-OS
	\item Linux
\end{todolist}
&
\begin{todolist}
	\item Windows
	\item macOS
\end{todolist}

\end{tabular}

\openup 1em

{\bf SMO products in scope:} \\

\openup -1em
\vspace{-3em}

\begin{tabular}{p{.5\textwidth}p{.5\textwidth}}

\begin{todolist}
  \item Generic
  \item VICOS OC 100
  \item VICOS S\&D
\end{todolist}
&
\begin{todolist}
  \item Rail9000
  \item[\done] DCS
  \item Others: Fortigate Firewall \hrulefill
\end{todolist}

\end{tabular}

\openup 1em

{\bf Description:} \\

\openup -1em
\vspace{-2em}

%\begin{itemize}
%	\item (What does the rule do?)
%	\item (Why was the situation targeted (e.g., in a workshop)?)
%\end{itemize}

This rule reports all security events from the unified threat management of the FortiGate firewall.

\openup 1em

{\bf Rule query:} \\

\openup -1em
\vspace{-2em}

\begin{lstlisting}[language=json,firstnumber=1]
{
  "query": {
    "bool": {
      "must": [
        {
          "bool": {
            "should": [
              {
                "match": {
                  "fortigateLevel": "emergency"
                }
              },
              {
                "match": {
                  "fortigateLevel": "alert"
                }
              },
              {
                "match": {
                  "fortigateLevel": "critical"
                }
              },
              {
                "match": {
                  "fortigateLevel": "error"
                }
              }
            ]
          }
        },
        {
          "match": {
            "fortigateType": "utm"
          }
        },
        {
          "range": {
            "@timestamp": {
              "gte": "<bt>",
              "lt": "<et>"
            }
          }
        }
      ]
    }
  },
  "size": 1000
}
\end{lstlisting}

\openup 1em

{\bf Rule requirements:} \\

\openup -1em
\vspace{-2em}

The FortiGate firewall sends syslogs to OSA and Logstash is configured appropriately.

\openup 1em

{\bf Rule parameters:} \\

\openup -1em
\vspace{-2em}

\begin{itemize}
	\item A standard time frame defined by the ``greater than'' (gt) and ``less than'' parameters.
\end{itemize}

\openup 1em

{\bf Further information/categorization:} \\

\openup -1em
\vspace{-2em}

%\begin{itemize}
%	\item <Does the rule have an ATT\&CK label?>
%	\item <Might the rule create many false positives?>
%\end{itemize}

None.

\pagebreak



%%%%%%%%%
%%% Others %%%
%%%%%%%%%

%\input{rules/Template}


\pagebreak

\chapter{SMO Workshops Results}
\markboth{}{SMO Workshops Results}
%\addcontentsline{toc}{chapter}{\protect\numberline{}SMO Workshops Results}%

The following tables present the results of the performed workshops with SMO product owners and technical experts. Each table shows the identified detection use case of interest and the corresponding OSA correlation rules (both old an new) used to implement the related detection mechanism.

% !TEX root = ./AdditionalOsaRules.tex

%%%
%
% Version 0.1
%
%%%

\section*{VICOS Products Family}
\addcontentsline{toc}{section}{\protect\numberline{}VICOS Products Family}%

The following table lists the detection use cases selected during the VICOS workshops (for more information, please, refer to the related Excel file). Every use case is associated with the corresponding OSA correlation rules.

\begin{center}
	\begin{longtable}{P{.3\textwidth}P{.6\textwidth}}
		\hline
		\textbf{Use Case} & \textbf{Rules} \\
		\hline
		Repeated Failed Login & \makecell[tl]{12020-Windows-Brute-Force \\ 15201-Repeated-Failed-Login~[\ref{15201}]} \\
		Account Lockout & \makecell[tl]{15017-Account-Locked-Out~[\ref{15017}] \\ 15202-Vicos-Account-Deletion*~[\ref{15202}]} \\
		Login of Non-Registered Accounts & \makecell[tl]{15021-Login-Of-Non-Registered-Account~[\ref{15021}] \\ 15203-Vicos-Login-Attempt-Of-Non-Authorized-Username~[\ref{15203}]} \\
		Login at Unusual Time & \makecell[tl]{15022-Login-Logout-At-Unusual-Time~[\ref{15022}] \\ 15204-Vicos-Login-Logout-At-Unusual-Time~[\ref{15204}]} \\
		File Share Client-Side Access & \makecell[tl]{00943-Network-Share-Access-on-Windows} \\
		File Share Server-Side Access & \makecell[tl]{00943-Network-Share-Access-on-Windows} \\
		Application Installation/Removal & \makecell[tl]{15023-Application-Installation~[\ref{15023}] \\ 15024-Application-Removal~[\ref{15024}] \\ 15025-Application-Installation-Applocker~[\ref{15025}]} \\
		New File Share & \makecell[tl]{15041-New-File-Share~[\ref{15041}]} \\
		New Directory/File & \makecell[tl]{15050-New-File-Or-Folder~[\ref{15050}]} \\
		Removed Directory/File & \makecell[tl]{15051-Deleted-File-Or-Folder~[\ref{15051}]} \\
		Access to Directory/File & \makecell[tl]{15052-Accessed-File-Or-Folder~[\ref{15052}]} \\
		Modification to Directory/File & \makecell[tl]{15053-Modified-File-Or-Folder~[\ref{15053}]} \\
		Malicious Process Execution & \makecell[tl]{15060-Malicious-Process-Execution~[\ref{15060}]} \\
		Anomalous Process Execution & \makecell[tl]{15061-Anomalous-Process-Execution~[\ref{15061}]} \\
		Anomalous Process Execution with High Privileges & \makecell[tl]{15062-Anomalous-Process-Execution-with-High-Privileges~[\ref{15062}]} \\
		Process Termination & \makecell[tl]{15063-Process-Termination~[\ref{15063}]} \\
		New Service & \makecell[tl]{15064-New-Service-Execution~[\ref{15064}]} \\
		New Hardware Plugged-in & \makecell[tl]{15030-New-Hardware-Plugged-In~[\ref{15030}]} \\
		New USB Storage Plugged-in & \makecell[tl]{15031-New-USB-Storage-Plugged-In~[\ref{15031}]} \\
		New USB Device Plugged-in & \makecell[tl]{15032-New-USB-Device-Plugged-In~[\ref{15032}]} \\
		OS User Account Creation & \makecell[tl]{00904-User-account-created-on-Windows} \\
		OS High-Privileged User Account Creatoin & \makecell[tl]{15026-Os-High-Privileged-Account-Creation~[\ref{15026}} \\
		Application High-Privileged Account Creation & \makecell[tl]{15205-Vicos-Critical-Account-Creation~[\ref{15205}]} \\
		Application Configuration Changed & \makecell[tl]{15027-Application-Configuration-Change~[\ref{15027}]} \\
		Operation Performed At Anomalous Time & \makecell[tl]{15206-Vicos-Operation-Performed-At-Anomalous-Time~[\ref{15206}]} \\
		Application Log Cleared & \makecell[tl]{15207-Vicos-Application-Log-Cleared~[\ref{15207}] \\ 15028-Window-Object-Deletion~[\ref{15028}]} \\
		Ineffective Security Produt & \makecell[tl]{15070-Windows-Applocker-Disabled~[\ref{15070}]} \\
		Security Product Alert & \makecell[tl]{15071-Windows-Applocker-Security-Event~[\ref{15071}]} \\
		System Log Space Full & \makecell[tl]{15080-Disk-Space-Exhaustion~[\ref{15080}]} \\
    		\hline
	\end{longtable}
\end{center}



% !TEX root = ./AdditionalOsaRules.tex

%%%
%
% Version 0.1
%
%%%

\section*{Rail9000 Products Family}
\addcontentsline{toc}{section}{\protect\numberline{}Rail9000 Products Family}%

The following table lists the detection use cases selected during the Rail9000 workshops (for more information, please, refer to the related Excel file). Every use case is associated with the corresponding OSA correlation rules.

\begin{center}
	\begin{longtable}{P{.3\textwidth}P{.6\textwidth}}
		\hline
		\textbf{Use Case} & \textbf{Rules} \\
		\hline
		Repeated Failed Login & \makecell[tl]{12020-Windows-Brute-Force \\ 15101-Repeated-Failed-Login~[\ref{15101}]} \\
		Access Unlocked & \makecell[tl]{15699-Access-Unlocked~[\ref{15699}]} \\
		Login at Unusual Time & \makecell[tl]{15698-Login-At-Unusual-Time-Linux~[\ref{15698}]} \\
		Anomalous Protocol & \makecell[tl]{``OSA Built-in Capability''} \\
		File Share Client-Side Access & \makecell[tl]{00943-Network-Share-Access-on-Windows} \\
		File Share Server-Side Access & \makecell[tl]{00943-Network-Share-Access-on-Windows} \\ 
		New Asset & \makecell[tl]{00201-Asset-Connected} \\
		Removed Asset & \makecell[tl]{00201-Asset-Disconnected} \\
		Network Topology Modification & \makecell[tl]{``OSA Built-in Capability''} \\
		Malicious Connection & \makecell[tl]{``OSA Built-in Capability'' \\ 12004-Telnetd-Debian-Connection \\ 08020-A-TCP-connection-was-detected \\ 01511-Vulnerable-Netlogon-Secure-Channel-Connection-Allowed} \\
		Anomalous Connection & \makecell[tl]{``OSA Built-in Capability''} \\
		Insecure Communication & \makecell[tl]{00853-Insecure-SSL-TLS-Version \\ 00854-Insecure-SSL-TLS-Cyphers} \\
		Application Installation/Removal & \makecell[tl]{15697-Application-Installation-Linux~[\ref{15697}] \\ 15696-Application-Removal-Linux~[\ref{15696}]} \\
		New File Share & \makecell[tl]{15041-New-File-Share~[\ref{15041}]} \\
		New Directory/File & \makecell[tl]{15651-Generic-File-Write~[\ref{15651}]} \\
		Removed Directory/File & \makecell[tl]{15653-Generic-File-Remove~[\ref{15653}]} \\
		Access to Directory/File & \makecell[tl]{15650-Generic-File-Read~[\ref{15650}]} \\
		Modification to Directory/File & \makecell[tl]{15651-Generic-File-Write~[\ref{15651}]} \\
		Malicious Process Execution & \makecell[tl]{15660-Malicious-Process-Execution~[\ref{15660}]} \\
		Anomalous Process Execution with High Privileges & \makecell[tl]{15662-Anomalous-Process-Execution-With-High-Privileges~[\ref{15662}]} \\
		New USB Input/Output Device Connected & \makecell[tl]{15695-New-Usb-Device-Plugged-In~[\ref{15695}]} \\
		OS User Account Creation & \makecell[tl]{15670-Os-User-Account-Creation-Linux~[\ref{15670}]} \\
		OS High-Privilege User Account Creation & \makecell[tl]{15671-Os-High-Privilege-User-Account-Creation-Linux~[\ref{15671}]} \\
		OS Temporary User Account & \makecell[tl]{15672-Os-Temporary-User-Account-Linux~[\ref{15672}]} \\
		System Log Cleared & \makecell[tl]{15653-Generic-File-Remove~[\ref{15653}]} \\
    		\hline
	\end{longtable}
\end{center}



% !TEX root = ./AdditionalOsaRules.tex

%%%
%
% Version 0.1
%
%%%

\section*{DCS-related Products}
\addcontentsline{toc}{section}{\protect\numberline{}DCS-related Products}%

The following table lists the detection use cases selected during the DCS workshops (for more information, please, refer to the related Excel file). Every use case is associated with the corresponding OSA correlation rules.

\begin{center}
	\begin{longtable}{P{.3\textwidth}P{.6\textwidth}}
		\hline
		\textbf{Use Case} & \textbf{Rules} \\
		\hline
		DCU-related Events & \makecell[tl]{15301-Dcu-Invalid-Signature~[\ref{15301}]} \\ 
		ESXi-related Events & \makecell[tl]{15401-Esxi-Authentication-Failure~[\ref{15401}] \\ 15402-Esxi-SSH-Enabled~[\ref{15402}]} \\ 
		Hirchmann-related Events & \makecell[tl]{15701-Login-Hirschmann-Switch~[\ref{15701}]} \\ 
		Fortigate-related Events & \makecell[tl]{15800-Firewall-Admin-Login~[\ref{15800}] \\ 15810-Firewall-UTM-Security-Event~[\ref{15810}]} \\ 
		Repeated Failed Login & \makecell[tl]{00920-Failed-Logon-Attempt-on-Windows \\ 11007-Brute-force-attempt-for-Samba-on-QNAP-NAS \\ 12001-Sshd-Brute-Force \\ 12003-Ftpd-Debian-Brute-Force \\ 12006-Vsftpd-Brute-Force \\ 12020-Windows-Brute-Force \\ 05085-SSH-Bruteforcing \\ 05086-FTP-Bruteforcing \\ 05087-TELNET-Bruteforcing \\ 05088-RDP-Bruteforcing} \\
		New Asset & \makecell[tl]{00201-Asset-connected} \\
		Removed Asset & \makecell[tl]{00209-Asset-disconnected} \\
		Modified Asset & \makecell[tl]{15053-Modified-File-Or-Folder~[\ref{15053}] \\ 00939-Registry-Modified-on-Windows \\ 01525-Unauthorized-System-Time-Modification} \\
		DoS & \makecell[tl]{00820-SYN-flood \\ 00864-ARP-flooding \\ 00869-UDP-flooding \\ 00870-DNS-query-flooding \\ 00871-SNMP-query-flooding \\ 05093-IEC-60870-5-104-Dos-Flooding-Attack} \\
		Newly-found Software Vulnerability & \makecell[tl]{05022-Vulnerability-Scan-CVE-2019-0708-Scan} \\
		System Log Cleared & \makecell[tl]{00936-Audit-Log-Cleared-on-Window} \\
		Account Lockout & \makecell[tl]{15017-Account-Locked-Out~[\ref{15017}]} \\
		Login Attempt of a Non-registered Account & \makecell[tl]{15021-Login-Of-Non-Registered-Account~[\ref{15021}]} \\
		Login at an Unusual Time & \makecell[tl]{15022-Login-Logout-At-Unusual-Time~[\ref{15302}]} \\
		File Share Client-side Access & \makecell[tl]{00943-Network-Share-Access-on-Windows} \\
		File Share Server-side Access & \makecell[tl]{00943-Network-Share-Access-on-Windows} \\
		Application Installation/Removal& \makecell[tl]{15023-Application-Installation~[\ref{15023}] \\ 15024-Application-Removal~[\ref{15024}]} \\% \\ 15025-Application-Installation-Applocker} \\
		New File Share & \makecell[tl]{15041-New-File-Share~[\ref{15041}]} \\
		New Directory/file & \makecell[tl]{15050-New-File-Or-Folder~[\ref{15050}]} \\
		Removed Directory/File & \makecell[tl]{15051-Deleted-File-Or-Folder~[\ref{15051}]} \\
		Access to Directory/File & \makecell[tl]{15052-Accessed-File-Or-Folder~[\ref{15052}]} \\
		Modification to Directory/File & \makecell[tl]{15053-Modified-File-Or-Folder~[\ref{15053}]} \\
		Malicious Process Execution & \makecell[tl]{15060-Malicious-Process-Execution~[\ref{15060}] \\ 01001-Wanna-Cry-Process} \\
		Anomalous Process Execution & \makecell[tl]{15061-Anomalous-Process-Execution~[\ref{15061}]} \\
		Anomalous Process Execution with High Privileges & \makecell[tl]{15062-Anomalous-Process-Execution-with-High-Privileges~[\ref{15062}]} \\
		Process Termination & \makecell[tl]{15063-Process-Termination~[\ref{15063}]} \\
		New Service/Task/Autorun & \makecell[tl]{15064-New-Service-Execution~[\ref{15064}]} \\
		New Hardware Plugged-in & \makecell[tl]{15030-New-Hardware-Plugged-In~[\ref{15030}]} \\
		New USB Storage Device was Connected & \makecell[tl]{15032-New-USB-Device-Plugged-In~[\ref{15032}] \\ 15031-New-USB-Storage-Plugged-In~[\ref{15031}]} \\
		OS Temporary User Account Creation & \makecell[tl]{00904-User-account-created-on-Windows} \\
		OS High-Privilege User Account Creation & \makecell[tl]{15026-Os-High-Privileged-Account-Creation~[\ref{15026}]} \\
		Application Log Cleared & \makecell[tl]{15028-Window-Object-Deletion~[\ref{15028}]} \\
		%Ineffective Security Product & \makecell[tl]{15070-Applocker-Disabled} \\
		%Security Product Alert & \makecell[tl]{15071-Applocker-Security-Event} \\
		Systen Log Soace Full & \makecell[tl]{15080-Disk-Space-Exhaustion~[\ref{15080}]} \\
		Anomalous Amount of DNS Requests/Responses & \makecell[tl]{00870-DNS-query-flooding} \\
		%Firewall Configuration Change & \makecell[tl]{00934-Firewall-Rule-Added-on-Windows \\ 00935-Firewall-Rule-Deleted-on-Windows} \\
		Malicious Connection & \makecell[tl]{``OSA Built-in Capability'' \\ 12004-Telnetd-Debian-Connection \\ 08020-A-TCP-connection-was-detected \\ 01511-Vulnerable-Netlogon-Secure-Channel-Connection-Allowed} \\
		Malicious Payload & \makecell[tl]{00891-SMBv1-DoublePulsar-Deliver-Payload} \\
		Insecure Communication & \makecell[tl]{00853-Insecure-SSL-TLS-Version \\ 00854-Insecure-SSL-TLS-Cyphers} \\
		Network Scan & \makecell[tl]{08023-Detection-of-a-Network-Scan \\ 00808-IP-Scan-(ICMP) \\ 05015-IP-Scan-(ARP) \\ 00815-Port-Scan-(Fast)} \\
		Modification of System Time & \makecell[tl]{01525-Unauthorized-System-Time-Modification} \\
		Missing Backups & \makecell[tl]{01528-Backup-Catalog-Deleted} \\
		Internet/Proxy Connection from System & \makecell[tl]{08037-Device-Retrieving-External-IP-Address-Detected} \\
		Use of Unusual Application/Tools & \makecell[tl]{00885-QQ-Wechat-Weibo-(Social-Media) \\ 00886-Baidu-Bing-Google-(Search-Engine)} \\
		TLS Certificate Problems & \makecell[tl]{00851-Expired-Certificate \\ 00850-Expiring-Certificate \\ 00849-Self-Signed-Certificate} \\
		SNMP Access & \makecell[tl]{00871-SNMP-query-flooding \\ 00872-Reflective-SNMP-DOS} \\
		% & \makecell[tl]{} \\
    		\hline
	\end{longtable}
\end{center}



\pagebreak

\chapter{Conclusion}
\markboth{}{Conclusion}
%\addcontentsline{toc}{chapter}{\protect\numberline{}Conclusion}%

All rules presented in this document have been made available to the OSA development team for integration. It has been agreed that the whole set of rules will be included in the OSA 4.6 installation package. Nonetheless, only the general rules without requirements will be switched on by default in the standard OSA 4.6 installation. This approach will avoid the indiscriminate use of rules created for specific environments (which should be deployed only with the operator confirming the presence of a matching monitored system).

\begin{appendices}
\chapter*{Appendix}
\label{appendix}

%\section*{List of ``countries of interest'' provided by DF}
%\label{appendix:df_countries}

\input{AppendixYMLTemplate.tex}
% !TEX root = ./AdditionalOsaRules.tex

%%%
%
% Version 0.1
%
%%%

\section*{Auditd Configuration}
\addcontentsline{toc}{section}{\protect\numberline{}Auditd Configuration}%
\label{appendix:auditdconfiguration}

\begin{lstlisting}[language=json,firstnumber=1]

## First rule - delete all
-D

## Increase the buffers to survive stress events.
## Make this bigger for busy systems
-b 8192

## This determine how long to wait in burst of events
--backlog_wait_time 60000

## Set failure mode to syslog
-f 1

## customized auditd rules
-w /usr/bin/passwd -p x -k binpasswd
-w /usr/sbin/useradd -p x -k usermodification
-w /usr/sbin/userdel -p x -k usermodification
-w /usr/sbin/usermod -p x -k usermodification
-w /usr/sbin/adduser -p x -k usermodification
-w /usr/sbin/groupadd -p x -k groupmodification
-w /usr/sbin/groupdel -p x -k groupmodiciation
-w /usr/sbin/groupmod -p x -k groupmodification
-w /usr/sbin/addgroup -p x -k groupmodification
-w /etc/group -p wa -k etcgroup
-w /etc/passwd -p wa -k etcpasswd
-w /etc/gshadow -k etcgroup
-w /etc/shadow -p wa -k etcshadow
-a exit,always -F arch=b64 -S adjtimex -S settimeofday -S clock_settime -k timechange

## generic auditd rules
# -w PATH_TO_FILE_X -p r -k file_x_read
# -w PATH_TO_FILE_X -p w -k file_x_write
# -w PATH_TO_FILE_X -p x -k file_x_execute

\end{lstlisting}

\pagebreak


% !TEX root = ./AdditionalOsaRules.tex

%%%
%
% Version 0.1
%
%%%

\section*{Auditd Logstash Configuration}
\addcontentsline{toc}{section}{\protect\numberline{}Auditd Logstash Configuration}%
\label{appendix:auditdlogstashconfiguration}

\begin{lstlisting}[language=json,firstnumber=1]

filter {
    if "deviceSyslog" in [tags] {
        if [program] == "tag_audit_log" { 
            mutate {
                gsub => ["msg","\u001D", " "]
            }
            kv {
                source => "msg"
                include_keys => ["type","key"]
            }
            mutate {
                rename => { "type" => "auditdType"}
                rename => { "key" => "auditdKey"}
            }
        }
    }    
}

\end{lstlisting}

\pagebreak


% !TEX root = ./AdditionalOsaRules.tex

%%%
%
% Version 0.1
%
%%%

\section*{Fortigate Logstash Configuration}
\addcontentsline{toc}{section}{\protect\numberline{}Fortigate Logstash Configuration}%
\label{appendix:fortigatelogstashconfiguration}

\begin{lstlisting}[language=json,firstnumber=1]

filter {
    if "deviceSyslog" in [tags] {
        if [program] == "fortigate" { 
            kv {
                source => "msg"
                include_keys => ["logid","type","subtype","level"]
            }
            mutate {
                rename => {"logid" => "fortigateLogid"}
                rename => {"type" => "fortigateType"}
                rename => {"subtype" => "fortigateSubtype"}
                rename => {"level" => "fortigateLevel"}
            }
        }
    }    
}

\end{lstlisting}

\pagebreak



\end{appendices}

%\bibliographystyle{abbrv}
%\bibliography{proposal}

\end{document}
