% !TEX root = ../AdditionalOsaRules.tex

%%%
%
% Version 0.1
%
%%%

\section{Auditd Generic File Remove}
\label{15653}
%\addcontentsline{toc}{section}{\protect\numberline{}(Tool)}%

\openup 1em

\textbf{Full name:} Auditd generic file remove\hrulefill \\
%\textbf{Vendor/Provider:} (Vendor name)\hrulefill \\
%\textbf{URL: } \url{http://www.google.de} \hrulefill \\
{\bf Categorization:} \\

\openup -1em
\vspace{-3em}

% Please, write either Multi-OS, Linux, Windows, Mac
\begin{tabular}{p{.5\textwidth}p{.5\textwidth}}

\begin{todolist}
  	\item Multi-OS
	\item[\done] Linux
\end{todolist}
&
\begin{todolist}
	\item Windows
	\item macOS
\end{todolist}

\end{tabular}

\openup 1em

{\bf SMO products in scope:} \\

\openup -1em
\vspace{-3em}

\begin{tabular}{p{.5\textwidth}p{.5\textwidth}}

\begin{todolist}
  \item[\done] Generic
  \item VICOS OC 100
  \item VICOS S\&D
\end{todolist}
&
\begin{todolist}
  \item[\done] Rail9000
  \item DCS
  \item Others: \hrulefill
\end{todolist}

\end{tabular}

\openup 1em

{\bf Description:} \\

\openup -1em
\vspace{-2em}

%\begin{itemize}
%	\item (What does the rule do?)
%	\item (Why was the situation targeted (e.g., in a workshop)?)
%\end{itemize}

This rule detects the removing action on a given file or directory.

\openup 1em

{\bf Rule query:} \\

\openup -1em
\vspace{-2em}

\begin{lstlisting}[language=json,firstnumber=1]
{
  "query": {
    "bool": {
      "must": [
        {
          "match": {
            "auditdType": "SYSCALL"
          }
        },
        {
          "match": {
            "auditdKey": "file_x_write"
          }
        },
        {
          "match_phrase": {
            "msg": "syscall=263"
          }
        },
        {
          "range": {
            "@timestamp": {
              "gte": "<bt>",
              "lt": "<et>"
            }
          }
        }
      ]
    }
  },
  "size": 1000
}
\end{lstlisting}

\openup 1em

{\bf Rule requirements:} \\

\openup -1em
\vspace{-2em}

Auditd should be configured (an example of a suitable configuration has been provided in the repository). Furthermore, Logstash should be configured to ingest the related data (also in this case, an example of a suitable configuration has been provided in the repository). A specific path should be included in the auditd rule identified by key ``file\_x\_write''.

\openup 1em

{\bf Rule parameters:} \\

\openup -1em
\vspace{-2em}

\begin{itemize}
	\item A standard time frame defined by the ``greater than'' (gt) and ``less than'' parameters.
\end{itemize}

\openup 1em

{\bf Further information/categorization:} \\

\openup -1em
\vspace{-2em}

%\begin{itemize}
%	\item <Does the rule have an ATT\&CK label?>
%	\item <Might the rule create many false positives?>
%\end{itemize}

The rule can be labeled with the following MITRE ATT\&CK's tactics and techniques:
\begin{itemize}
	\item T1485 - ``Data Destruction''
	\item T1565 - ``Data Manipulation''
\end{itemize}

\pagebreak

